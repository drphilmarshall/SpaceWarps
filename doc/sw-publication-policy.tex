%%%%%%%%%%%%%%%%%%%%%%%%%%%%%%%%%%%%%%%%%%%%%%%%%%%%%%%%%%%%%%%%%%%%%%%%
%
% Space Warps Publication Policy
%
%%%%%%%%%%%%%%%%%%%%%%%%%%%%%%%%%%%%%%%%%%%%%%%%%%%%%%%%%%%%%%%%%%%%%%%%

\documentclass[a4paper]{article}

% Packages:
\usepackage{xspace}
\usepackage{verbatim}

% Macros:
% JOURNALS
\newcommand{\apj}{ApJ}
\newcommand{\apjl}{ApJL}
\newcommand{\apjs}{ApJS}
\newcommand{\mnras}{MNRAS}
\newcommand{\apss}{Ap \& SS}
\newcommand{\aap}{A\&A}
\newcommand{\aj}{AJ}
\newcommand{\prd}{Phys. Rev. D}
\newcommand{\nat}{Nature}
\newcommand{\araa}{ARA\&A}
\newcommand{\jgr}{J. Geophys. Res.}
\newcommand{\pasp}{PASP}

% MISC
\newcommand{\etal}{et~al.~}
\newcommand{\eg}{{\it e.g.\ }}
\newcommand{\ie}{{\it i.e.\ }}
\newcommand{\etc}{{\it etc.\ }}

\newcommand{\be}{\begin{equation}}
\newcommand{\ee}{\end{equation}}
\newcommand{\bea}{\begin{eqnarray}}
\newcommand{\eea}{\end{eqnarray}}


% CROSS-REFERENCING
\def\Sref#1{Section~\ref{#1}\xspace}
\def\Fref#1{Figure~\ref{#1}\xspace}
\def\Tref#1{Table~\ref{#1}\xspace}
\def\Eref#1{Equation~\ref{#1}\xspace}
\def\Aref#1{Appendix~\ref{#1}\xspace}

\newcommand{\PaperOne}{Paper~I\xspace}
\newcommand{\PaperTwo}{Paper~II\xspace}

\newcommand{\StageOne}{Stage~1\xspace}
\newcommand{\StageTwo}{Stage~2\xspace}

% UNITS
\newcommand{\kms}{\ifmmode  \,\rm km\,s^{-1} \else $\,\rm km\,s^{-1}  $
\fi }
\newcommand{\kpc}{\ifmmode  {\rm kpc}  \else ${\rm  kpc}$ \fi  }
\newcommand{\pc}{\ifmmode  {\rm pc}  \else ${\rm pc}$ \fi  }
\newcommand{\Msun}{\ifmmode {\rm M_{\odot}} \else ${\rm M_{\odot}}$ \fi}
\newcommand{\Zsun}{\ifmmode {\rm Z_{\odot}} \else ${\rm Z_{\odot}}$ \fi}
\newcommand{\yr}{\ifmmode yr^{-1} \else $yr^{-1}$ \fi}
\newcommand{\hMsun}{\ifmmode h^{-1}\,\rm M_{\odot} \else $h^{-1}\,\rm
M_{\odot}$ \fi}

% COSMOLOGY
\newcommand{\LCDM}{$\Lambda{\rm CDM}$}
\newcommand{\MS}{Millennium Simulation\xspace}

% LENSING
\def\zd{z_{\rm d}}
\def\zs{z_{\rm s}}
\def\Dd{D_{\rm d}}
\def\Ds{D_{\rm s}}
\def\Dt{D_{\Delta t}}
\def\Dds{D_{\rm ds}}
\def\Sigmacrit{\Sigma_{\rm crit}}
\def\REin{R_{\rm Ein}}
\def\MEin{M_{\rm Ein}}

% SOFTWARE/HARDWARE
\def\sw{{\small\sc Space\,Warps}\xspace}
\def\SW{{\sc Space\,Warps}\xspace}
\def\Talk{{\small\sc Talk}\xspace}
\def\Letters{{\small\sc Letters}\xspace}
\def\Letter{{\small\sc Letter}\xspace}
\def\Dashboard{{\small\sc Dashboard}\xspace}
\def\cfhtls{{CFHTLS}\xspace}
\def\python{{\sc python}\xspace}
\def\gravlens{{\sc gravlens}\xspace}
\def\sextractor{{\sc SExtractor}\xspace}
\def\fitsjs{{\sc fitsjs}\xspace}
\def\humvi{{\sc HumVI}\xspace}
\def\af{{\sc ArcFinder}\xspace}
\def\AF{{\sc ArcFinder}\xspace}
\def\rf{{\sc RingFinder}\xspace}
\def\RF{{\sc RingFinder}\xspace}
\def\PH{{\sc Planet\,Hunters}\xspace}
\def\GZ{{\sc Galaxy\,Zoo}\xspace}

% TABLES:
\newcommand\nodata{ ~$\cdots$~ }%

% PROBABILITY THEORY
\def\pr{{\rm Pr}}
\def\data{{\mathbf{d}}}
\def\datap{{\mathbf{d}^{\rm p}}}
\def\training{{\mathbf{d}^{\rm t}}}
\def\trainingk{{\mathbf{d}^{\rm t}_k}}
\def\datai{d_i}
\def\datapi{d^{\rm p}_i}
\def\LENS{{\rm LENS}}
\def\saidLENS{{\rm ``LENS"}}
\def\NOT{{\rm NOT}}
\def\saidNOT{{\rm ``NOT"}}
\def\CM{\mathcal{M}}
\def\PL{\CM_{LL}}
\def\PD{\CM_{NN}}

% AGENT/SAMPLE BUREAUCRACY
\def\effort{N_{\rm C}}
\def\thiseffort{N_{{\rm C},k}}
\def\experience{N_{\rm T}}
\def\skill{{\langle \Delta I \rangle_{0.5}}}
\def\contribution{\skill^{\rm total}}
\def\information{\Delta I}
\def\Ns{J} % Number of subjects
\def\Nv{K} % Number of volunteers
\def\Ncands{N_{\rm det}} % Number of detected candidates

% COMMENTING
\usepackage[usenames]{color}
\newcommand{\question}[2]{\textcolor{red}{Question from #1: #2}}
\newcommand{\flag}[2]{\textcolor{red}{#1: #2}}
\newcommand{\todo}[2]{\textcolor{red}{\bf To Do: #1: #2}}
\newcommand{\new}[1]{\textcolor{blue}{#1}}


%%%%%%%%%%%%%%%%%%%%%%%%%%%%%%%%%%%%%%%%%%%%%%%%%%%%%%%%%%%%%%%%%%%%%%%%

\begin{document}
             
\title{\SW Collaboration Publication Policy}
\author{Phil Marshall}
\date{April 15, 2013}
\maketitle

%%%%%%%%%%%%%%%%%%%%%%%%%%%%%%%%%%%%%%%%%%%%%%%%%%%%%%%%%%%%%%%%%%%%%%%%

\begin{abstract} 
\noindent In this document we define the various types of \sw
collaboration membership, and their meaning in terms of the scientific
papers that \sw contributes to. \sw was conceived as a strong  lens
discovery {\it service}, that any survey science team could make use of
(within the logistical constraints). Here, we discuss the relationship
between the \sw collaboration and the survey teams, and suggest
reasonable guidelines for publication of \sw-enabled discoveries.
\end{abstract}

\setcounter{footnote}{1}

%%%%%%%%%%%%%%%%%%%%%%%%%%%%%%%%%%%%%%%%%%%%%%%%%%%%%%%%%%%%%%%%%%%%%%%%%%%%%%

\section{Membership Types}
\label{sec:members}

The \sw website is designed to motivate and  enable tens of thousands of
people to perform the scientific tasks of strong gravitational lens
identification and classification.  The \sw collaboration comprises the
sum total of the volunteers who have logged in and contributed
classifications to the project. (We would include the ones who did not
log in as well, but we don't know who they are.) This means that the
minimal requirement for \sw collaboration membership is a Zooniverse ID,
and some contribution on the site (\ie classifying, or discussing
Subjects in \Talk).

Not all collaboration members will participate in writing journal
papers. Those who have made significant contributions to the site
design, construction and operation, or in data preparation or analysis,
or in writing and editing the papers, will be recognised on the author
lists of journal papers presenting \sw results. The current list of \sw 
authors is given in the \texttt{authors.tex} file and is reprinted
below; this file as shown below is as it is currently configured, ready
to be copied and pasted into the system paper (\sw I).

{\small\verbatiminput{authors.tex}}

\noindent The author categories referred to above form the following
nested set:  

\begin{description}

\item{\bf Principals:} These are the PIs of the \sw project (Marshall, 
A.\,More and Verma), and the Zooniverse development team (led by
Kapadia).

\item{\bf Science Team:} This is a core group of scientists, both citizen and
professional, that have helped design, build and test the \sw website
and its initial project. The Science Team includes the PIs. Each member
of the Science Team will be listed on the website with a one paragraph
biography, and their names will appear in the \Talk site listed as
``Science Team.''

\item{\bf Other Authors:} This group makes up the remainder of the author
list, and includes people who have made only minor contributions to the
project. Such contributions might include helping write the initial
Citizen Science Alliance proposal, participating in discussions on
\Talk, writing \sw blog posts, advising on the data analysis, or editing
papers.

\end{description}

\noindent The author list can only grow, not shrink, over time. Within
th eauthor list, movement of, for example, one of the ``other authors''
into the ``Science Team'' could be readily achieved by that author
making a substantial contribution to the project (by, for example,
extensive participation in \Talk during the CFHTLS project). The author
list will be maintained and administered by the \sw PIs.

In the remaining sections we look at the various publications that might
arise from the \sw project, and suggest reasonable guidelines for
deciding on their authorship. First, however, we remind ourselves about data
access via \sw.

%%%%%%%%%%%%%%%%%%%%%%%%%%%%%%%%%%%%%%%%%%%%%%%%%%%%%%%%%%%%%%%%%%%%%%%%%%%%%%

\section{Survey Data Use}
\label{sec:data}

\sw is a public website. All images displayed there are by definition in the
public domain, and so must be expected to be downloaded, copied and
redistributed by any \sw user. This is a Good Thing: images of the sky taken
with publically-funded telescopes belong to everyone. It is the responsibility
of the survey science teams to provide images in a way that is consistent
with their own data access rules. There are two things that survey teams can
do to in order to respect any proprietary period that they have imposed on
themselves:
\begin{enumerate}

\item Add a ``LICENSE'' keyword to the FITS and PNG image headers, explaining
what the rules for redistribution of these images are. This will almost
certainly be ignored, but it would be a nice reminder that nothing comes free
of either cost or responsibility. Other keywords could also be included as
well: data provenance is important, and links to useful survey webpages would
be most welcome!

\item Remove all WCS information from all images provided. This would prevent
follow-up of any of the objects in the survey by anyone other than the survey
team. In practice, the objects contained in the \sw images will be too faint
for anyone without access to a large telescope to observe, and will also be
absent from any public catalogs, so the opportunities for follow-up will be
quite limited. Removing the WCS information is an insurance policy against any
competing professional scientists looking to ``scoop'' the survey team,
despite the damage it would do to their reputations. We note that images
with field of view less than about 3~arcminutes in diameter are not solvable
by \texttt{astrometry.net}.

\end{enumerate}

Hopefully, the benefits of using the \sw service will far outweigh the small
inconvenience represented by the above suggestions: the amount of other work
required to prepare an \sw dataset is much greater than this, in any case!


%%%%%%%%%%%%%%%%%%%%%%%%%%%%%%%%%%%%%%%%%%%%%%%%%%%%%%%%%%%%%%%%%%%%%%%%%%%%%%

\section{Zooniverse Publications: Letters and Talk }
\label{sec:series}

Any \sw collaboration member, simply by virtue of their Zooniverse
registration, may write a Zooniverse
Letter\footnote{\texttt{http://letters.zooniverse.org}} describing their
investigation of any lens candidate they find in \sw. This, along with
posts in  \Talk, is the primary means by which we expect collaboration
members will publish results of any detailed analysis to the rest of the
astronomical community. The investigation of any \sw images that are
provided without world coordinate system (WCS) information will be
necessarily limited, but the survey scientists who own this proprietary
information will benefit from the understanding gained in the \sw
\Letters, and science will proceed: Zooniverse \Letters are citable
objects, and in some cases will appear listed on ADS. 


%%%%%%%%%%%%%%%%%%%%%%%%%%%%%%%%%%%%%%%%%%%%%%%%%%%%%%%%%%%%%%%%%%%%%%%%%%%%%%

\section{The \SW Paper Series}
\label{sec:series}

We are planning a short series of papers describing the \sw service, its
first dataset (CFHTLS), and results from its investigation by the
collaboration. These results will likely include a sample of new lenses,
and some comparison studies between the visual and automated
identification of lens systems in this survey. This series will be known
as ``the \sw papers,'' and each one will have a title that starts with ``\sw.''

The Science Team and the Principals will be authors of these papers by
default, but will of course have the right to opt out if they wish. The
other authors may opt in to the author list of any of the \sw papers if
they feel that they have made a substantial contribution to that
investigation. The other authors may also be added to the author list of
any \sw paper; if this happens they will still have (of course) the
right to opt out again.

Beyond these rules, the first author of any given paper will decide, as
usual, on the author list of their own paper. This will include inviting
scientists from outside the list of authors, or from outside the
collaboration. The PIs will act as adjudicators if any disputes arise,
but to be honest we are not expecting any.


%%%%%%%%%%%%%%%%%%%%%%%%%%%%%%%%%%%%%%%%%%%%%%%%%%%%%%%%%%%%%%%%%%%%%%%%%%%%%%

\section{Future Survey Papers}
\label{sec:series}

The \sw papers must, by definition, provide a complete description of
the system, and its results from the CFHTLS survey. The science team of
any other survey (such as DES, RCS, KIDS, PS1, HSC \etc) should be able
to read these papers before designing their own \sw project, and must be
able to cite them as justification for some of their decisions.

Implementing a new project based on different survey data will,
nonetheless, require support from the \sw Science Team, in three
respects: 
\begin{enumerate}

\item In {\bf preparing data} for the site, and uploading it onto the
Zooniverse servers. The \sw Science Team can advise on suitable Training
Subjects and image display settings, and also on formatting the data
ready for ingestion and display 

\item In {\bf reconfiguring the site itself} ready for the new data. The
survey team will need to include new Spotter's Guide images and text,
modified tutorial content, and additional survey-specific site content;
the \sw Science Team can help with all of this.

\item In {\bf maintaining the site} so that it continues to function
correctly. This will involve technical support from the Zooniverse team,
but will also require the continued engagement of the wider
collaboration by the Science Team in \Talk. Since this social filter is
a key part of the sample refinement, the latter may involve a
significant time commitment.

\end{enumerate}

\noindent Given the above, we suggest that the PIs be able to nominate a
small number of \sw collaboration members as authors on any paper that
presents a lens discovery enabled by the \sw system. We do not yet know
how many people this would be, but it would likely be a fairly small
number (less than ten). All follow-up papers would of course be exempt
from this: the idea is that \sw is purely a gravitational lens discovery
service, but not one that runs by magic.


%%%%%%%%%%%%%%%%%%%%%%%%%%%%%%%%%%%%%%%%%%%%%%%%%%%%%%%%%%%%%%%%%%%%%%%%%%%%%%

\end{document}

%%%%%%%%%%%%%%%%%%%%%%%%%%%%%%%%%%%%%%%%%%%%%%%%%%%%%%%%%%%%%%%%%%%%%%%%%%%%%%
