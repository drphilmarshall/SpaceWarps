%%%%%%%%%%%%%%%%%%%%%%%%%%%%%%%%%%%%%%%%%%%%%%%%%%%%%%%%%%%%%%%%%%%%%%%%
%
% Space Warps I: Experiment Design
%
%%%%%%%%%%%%%%%%%%%%%%%%%%%%%%%%%%%%%%%%%%%%%%%%%%%%%%%%%%%%%%%%%%%%%%%%

\documentclass[useAMS,usenatbib,a4paper]{mn2e}
%% letterpaper
%% a4paper

\voffset=-0.6in

% Packages:
\input psfig.sty
\usepackage{xspace}
\usepackage{graphicx}
\usepackage{amssymb}
\usepackage{amsmath}

% Macros:
% JOURNALS
\newcommand{\apj}{ApJ}
\newcommand{\apjl}{ApJL}
\newcommand{\apjs}{ApJS}
\newcommand{\mnras}{MNRAS}
\newcommand{\apss}{Ap \& SS}
\newcommand{\aap}{A\&A}
\newcommand{\aj}{AJ}
\newcommand{\prd}{Phys. Rev. D}
\newcommand{\nat}{Nature}
\newcommand{\araa}{ARA\&A}
\newcommand{\jgr}{J. Geophys. Res.}
\newcommand{\pasp}{PASP}

% MISC
\newcommand{\etal}{et~al.~}
\newcommand{\eg}{{\it e.g.\ }}
\newcommand{\ie}{{\it i.e.\ }}
\newcommand{\etc}{{\it etc.\ }}

\newcommand{\be}{\begin{equation}}
\newcommand{\ee}{\end{equation}}
\newcommand{\bea}{\begin{eqnarray}}
\newcommand{\eea}{\end{eqnarray}}


% CROSS-REFERENCING
\def\Sref#1{Section~\ref{#1}\xspace}
\def\Fref#1{Figure~\ref{#1}\xspace}
\def\Tref#1{Table~\ref{#1}\xspace}
\def\Eref#1{Equation~\ref{#1}\xspace}
\def\Aref#1{Appendix~\ref{#1}\xspace}

\newcommand{\PaperOne}{Paper~I\xspace}
\newcommand{\PaperTwo}{Paper~II\xspace}

\newcommand{\StageOne}{Stage~1\xspace}
\newcommand{\StageTwo}{Stage~2\xspace}

% UNITS
\newcommand{\kms}{\ifmmode  \,\rm km\,s^{-1} \else $\,\rm km\,s^{-1}  $
\fi }
\newcommand{\kpc}{\ifmmode  {\rm kpc}  \else ${\rm  kpc}$ \fi  }
\newcommand{\pc}{\ifmmode  {\rm pc}  \else ${\rm pc}$ \fi  }
\newcommand{\Msun}{\ifmmode {\rm M_{\odot}} \else ${\rm M_{\odot}}$ \fi}
\newcommand{\Zsun}{\ifmmode {\rm Z_{\odot}} \else ${\rm Z_{\odot}}$ \fi}
\newcommand{\yr}{\ifmmode yr^{-1} \else $yr^{-1}$ \fi}
\newcommand{\hMsun}{\ifmmode h^{-1}\,\rm M_{\odot} \else $h^{-1}\,\rm
M_{\odot}$ \fi}

% COSMOLOGY
\newcommand{\LCDM}{$\Lambda{\rm CDM}$}
\newcommand{\MS}{Millennium Simulation\xspace}

% LENSING
\def\zd{z_{\rm d}}
\def\zs{z_{\rm s}}
\def\Dd{D_{\rm d}}
\def\Ds{D_{\rm s}}
\def\Dt{D_{\Delta t}}
\def\Dds{D_{\rm ds}}
\def\Sigmacrit{\Sigma_{\rm crit}}
\def\REin{R_{\rm Ein}}
\def\MEin{M_{\rm Ein}}

% SOFTWARE/HARDWARE
\def\sw{{\small\sc Space\,Warps}\xspace}
\def\SW{{\sc Space\,Warps}\xspace}
\def\Talk{{\small\sc Talk}\xspace}
\def\Letters{{\small\sc Letters}\xspace}
\def\Letter{{\small\sc Letter}\xspace}
\def\Dashboard{{\small\sc Dashboard}\xspace}
\def\cfhtls{{CFHTLS}\xspace}
\def\python{{\sc python}\xspace}
\def\gravlens{{\sc gravlens}\xspace}
\def\sextractor{{\sc SExtractor}\xspace}
\def\fitsjs{{\sc fitsjs}\xspace}
\def\humvi{{\sc HumVI}\xspace}
\def\af{{\sc ArcFinder}\xspace}
\def\AF{{\sc ArcFinder}\xspace}
\def\rf{{\sc RingFinder}\xspace}
\def\RF{{\sc RingFinder}\xspace}
\def\PH{{\sc Planet\,Hunters}\xspace}
\def\GZ{{\sc Galaxy\,Zoo}\xspace}

% TABLES:
\newcommand\nodata{ ~$\cdots$~ }%

% PROBABILITY THEORY
\def\pr{{\rm Pr}}
\def\data{{\mathbf{d}}}
\def\datap{{\mathbf{d}^{\rm p}}}
\def\training{{\mathbf{d}^{\rm t}}}
\def\trainingk{{\mathbf{d}^{\rm t}_k}}
\def\datai{d_i}
\def\datapi{d^{\rm p}_i}
\def\LENS{{\rm LENS}}
\def\saidLENS{{\rm ``LENS"}}
\def\NOT{{\rm NOT}}
\def\saidNOT{{\rm ``NOT"}}
\def\CM{\mathcal{M}}
\def\PL{\CM_{LL}}
\def\PD{\CM_{NN}}

% AGENT/SAMPLE BUREAUCRACY
\def\effort{N_{\rm C}}
\def\thiseffort{N_{{\rm C},k}}
\def\experience{N_{\rm T}}
\def\skill{{\langle \Delta I \rangle_{0.5}}}
\def\contribution{\skill^{\rm total}}
\def\information{\Delta I}
\def\Ns{J} % Number of subjects
\def\Nv{K} % Number of volunteers
\def\Ncands{N_{\rm det}} % Number of detected candidates

% COMMENTING
\usepackage[usenames]{color}
\newcommand{\question}[2]{\textcolor{red}{Question from #1: #2}}
\newcommand{\flag}[2]{\textcolor{red}{#1: #2}}
\newcommand{\todo}[2]{\textcolor{red}{\bf To Do: #1: #2}}
\newcommand{\new}[1]{\textcolor{blue}{#1}}

\def\oxford{Dept.\ of Physics, University of Oxford, Keble Road, Oxford, OX1 3RH, UK}
\def\kipac{Kavli Institute for Particle Astrophysics and Cosmology, Stanford University, 452 Lomita Mall, Stanford, CA 94035, USA}
\def\ipmu{Kavli IPMU (WPI), University of Tokyo, 5-1-5 Kashiwanoha, Kashiwa 277-8583, Japan}
\def\adler{Adler Planetarium, Chicago, IL, USA}
\def\lausanne{EPFL, Lausanne, Switzerland}
\def\zurich{Department of Physics, University of Zurich, Switzerland}

\def\pjmemail{\tt pjm@slac.stanford.edu}
\def\amemail{\tt anupreeta.more@ipmu.jp}


%%%%%%%%%%%%%%%%%%%%%%%%%%%%%%%%%%%%%%%%%%%%%%%%%%%%%%%%%%%%%%%%%%%%%%%%

\title[\sw I]
{\SW: Crowd-sourcing the Discovery of Gravitational Lenses. 
I: Experiment Design, and Testing with Simulated Data}
    
\author[The \SW Collaboration]{%
  The \SW Collaboration:
  Phil~Marshall,$^{1}$\thanks{\pjmemail}
  Aprajita Verma,$^{1}$
  Anupreeta More,$^{2}$
\newauthor{%
  Amit Kapadia,$^{3}$
  David Miller,$^{3}$
  Kelly Borden,$^{3}$
  Arfon Smith,$^{3}$}
\newauthor{%
  Elisabeth Baeten,
  Claude Cornen,
  Cecile Faure,
  Thomas Jennings,
  Rafael Kueng,$^{4}$}
\newauthor{%
  Christine Macmillan,
  Surhud More,$^{2}$
  Prasenjit Saha,$^{4}$
  Julianne Wilcox,
  Layne Wright}
  \medskip\\
  $^1$\oxford\\
  $^2$\ipmu\\
  $^3$\adler\\
  $^4$\zurich
}

% This list is incomplete.

%%%%%%%%%%%%%%%%%%%%%%%%%%%%%%%%%%%%%%%%%%%%%%%%%%%%%%%%%%%%%%%%%%%%%%%%

\begin{document}
             
\date{to be submitted to MNRAS? Not necessarily.}
\pagerange{\pageref{firstpage}--\pageref{lastpage}}\pubyear{2012}

\maketitle           

\label{firstpage}

%%%%%%%%%%%%%%%%%%%%%%%%%%%%%%%%%%%%%%%%%%%%%%%%%%%%%%%%%%%%%%%%%%%%%%%%

\begin{abstract} 


\end{abstract}

% Full list of options at http://www.journals.uchicago.edu/ApJ/instruct.key.html

\begin{keywords}
  gravitational lensing   --
  methods: statistical    
\end{keywords}

\setcounter{footnote}{1}

%%%%%%%%%%%%%%%%%%%%%%%%%%%%%%%%%%%%%%%%%%%%%%%%%%%%%%%%%%%%%%%%%%%%%%%%%%%%%%

\section{Introduction}
\label{sec:intro}

Scientific motivation. Applications of lenses: group-scale arcs,
galaxy-galaxy lenses, lensed quasars. 

Problem of rarity. Imaging surveys. Problem of purity/false positives.

Review of progress to date. Methods in SL2S, SQLS. Contrast with SLACS. 

Scaling to wide field era. Automated methods: problems. Need for good
training sets. Need for quality control: always present.

Novel solution: crowd-sourcing. Brief review of similar problems.
PlanetHunters. \sw as an experiment.

In this paper, we describe the \sw website, an online system that
enables crowd-sourced detection of gravitational lenses. 
Other papers in this series will present new gravitational lenses
discovered in our first imaging survey dataset, and investigate the
differences between lens detections made in \sw and those made with
automated techniques. Here, we try to answer the following questons:
\begin{itemize}

\item How reliably can we find gravitational lenses using the \sw
system? What is the completeness of the sample produced?

\item How noisy is the system? What is the purity of the sample
produced?

\item How accurately can gravitational lenses be located on the sky
during the identification process?

\item How quickly can lenses be detected? How many classifiers are
needed per target, and how quickly can they mobilise?

\end{itemize}

In \Sref{sec:overview} we...



%%%%%%%%%%%%%%%%%%%%%%%%%%%%%%%%%%%%%%%%%%%%%%%%%%%%%%%%%%%%%%%%%%%%%%%%%%%%%%

\section{Experiment Design}
\label{sec:design}


%%%%%%%%%%%%%%%%%%%%%%%%%%%%%%%%%%%%%%%%%%%%%%%%%%%%%%%%%%%%%%%%%%%%%%%%%%%%%%

\section{Data}
\label{sec:data}

Training data and test data. Refer to Paper II for CFHTLS details.

Presentation of images.


%%%%%%%%%%%%%%%%%%%%%%%%%%%%%%%%%%%%%%%%%%%%%%%%%%%%%%%%%%%%%%%%%%%%%%%%%%%%%%

\section{Analysis}
\label{sec:analysis}

Interpreting clicks. Online analysis: update model of each volunteer,
and estimate probability of classification.

Training data for tracking learning, and estimating completeness and
purity.

%%%%%%%%%%%%%%%%%%%%%%%%%%%%%%%%%%%%%%%%%%%%%%%%%%%%%%%%%%%%%%%%%%%%%%%%%%%%%%

\section{Results}
\label{sec:results}

Understanding crowd, so we can help them learn faster. Understanding
images given the crowd, so we can find lenses.

% % % % % % % % % % % % % % % % % % % % % % % % % % % % % % % % % % % % % % % 

\subsection{Learned Behavior}
\label{sec:results:learning}

% % % % % % % % % % % % % % % % % % % % % % % % % % % % % % % % % % % % % % % 

\subsection{Sample completeness and purity}
\label{sec:results:learning}



%%%%%%%%%%%%%%%%%%%%%%%%%%%%%%%%%%%%%%%%%%%%%%%%%%%%%%%%%%%%%%%%%%%%%%%%%%%%%%

\section{Discussion}
\label{sec:discuss}

Challenges for future.

%%%%%%%%%%%%%%%%%%%%%%%%%%%%%%%%%%%%%%%%%%%%%%%%%%%%%%%%%%%%%%%%%%%%%%%%%%%%%%

\section{Conclusions}
\label{sec:conclude}

We draw the following conclusions:

\begin{itemize} 

\item Ability to detect lenses

\item What we can say about the next steps

\end{itemize}

Summarize, end.

%%%%%%%%%%%%%%%%%%%%%%%%%%%%%%%%%%%%%%%%%%%%%%%%%%%%%%%%%%%%%%%%%%%%%%%%
%%  ACKNOWLEDGMENTS
%%%%%%%%%%%%%%%%%%%%%%%%%%%%%%%%%%%%%%%%%%%%%%%%%%%%%%%%%%%%%%%%%%%%%%%%

\section*{Acknowledgements}
 
We thank all \Ncollaboration members of the \sw community for their
contributions to the project so far. A complete list of collaborators is
given at... In particular we would like to recognise the efforts of XXX,
YYY etc in moderating the discussion.

We are also grateful to Brooke Simmons, David Hogg, XXX and YYY for many useful
conversations about citizen science and gravitational lens detection. 
%
PJM was given support by the Royal Society, in the form of a research
fellowship, and by the U.S. Department of Energy under contract number DE-AC02-76SF00515.
%
AV acknowledges support from the Leverhulme Trust in the form of a research
fellowship.
%
The work of AM and SM was supported by World Premier International Research
Center Initiative (WPI Initiative), MEXT, Japan.
%
% Zooniverse / Sloan Foundation.
% 
% Other authors.
PJM and ES thank the Institute of Astronomy and Astrophysics, Academia Sinica
(ASIAA) and Taiwan's Ministry of Science and Technology (MOST) for their
financial support of the workshop ``Citizen Science in Astronomy'' in March
2014, at which some parts of the SWAP analysis was developed.


The \sw project is open source. 
The web app was developed at https://github.com/Zooniverse/Lens-Zoo
while the SWAP analysis software was developed at
https://github.com/drphilmarshall/SpaceWarps.


%%%%%%%%%%%%%%%%%%%%%%%%%%%%%%%%%%%%%%%%%%%%%%%%%%%%%%%%%%%%%%%%%%%%%%%%%%%%%%
%  APPENDICES
%%%%%%%%%%%%%%%%%%%%%%%%%%%%%%%%%%%%%%%%%%%%%%%%%%%%%%%%%%%%%%%%%%%%%%%%%%%%%%

\appendix

%%%%%%%%%%%%%%%%%%%%%%%%%%%%%%%%%%%%%%%%%%%%%%%%%%%%%%%%%%%%%%%%%%%%%%%%%%%%%%

\section{Probabilistic Classification Model}
\label{appendix:probmodel}


%%%%%%%%%%%%%%%%%%%%%%%%%%%%%%%%%%%%%%%%%%%%%%%%%%%%%%%%%%%%%%%%%%%%%%%%%%%%%%
%  REFERENCES
%%%%%%%%%%%%%%%%%%%%%%%%%%%%%%%%%%%%%%%%%%%%%%%%%%%%%%%%%%%%%%%%%%%%%%%%%%%%%%

% MNRAS does not use bibtex, input .bbl file instead. 
% Generate this in the makefile using bubble script in scriptutils:

% bubble -f paper-lcr.tex references.bib 
% \input{paper-lcr.bbl}

\bibliographystyle{apj}
\bibliography{references}


%%%%%%%%%%%%%%%%%%%%%%%%%%%%%%%%%%%%%%%%%%%%%%%%%%%%%%%%%%%%%%%%%%%%%%%%%%%%%%

\label{lastpage}
\bsp

\end{document}

%%%%%%%%%%%%%%%%%%%%%%%%%%%%%%%%%%%%%%%%%%%%%%%%%%%%%%%%%%%%%%%%%%%%%%%%%%%%%%
