%%%%%%%%%%%%%%%%%%%%%%%%%%%%%%%%%%%%%%%%%%%%%%%%%%%%%%%%%%%%%%%%%%%%%%%%%%%%%%
%
% Space Warps CFHTLS Results Paper
%
%%%%%%%%%%%%%%%%%%%%%%%%%%%%%%%%%%%%%%%%%%%%%%%%%%%%%%%%%%%%%%%%%%%%%%%%%%%%%%

\documentclass[useAMS,usenatbib,a4paper]{mn2e}
%% letterpaper
%% a4paper

\voffset=-0.6in

% Packages:
\input psfig.sty
\usepackage{xspace}
\usepackage{graphicx}
\usepackage{amssymb}
\usepackage{amsmath}

% Macros:
% JOURNALS
\newcommand{\apj}{ApJ}
\newcommand{\apjl}{ApJL}
\newcommand{\apjs}{ApJS}
\newcommand{\mnras}{MNRAS}
\newcommand{\apss}{Ap \& SS}
\newcommand{\aap}{A\&A}
\newcommand{\aj}{AJ}
\newcommand{\prd}{Phys. Rev. D}
\newcommand{\nat}{Nature}
\newcommand{\araa}{ARA\&A}
\newcommand{\jgr}{J. Geophys. Res.}
\newcommand{\pasp}{PASP}

% MISC
\newcommand{\etal}{et~al.~}
\newcommand{\eg}{{\it e.g.\ }}
\newcommand{\ie}{{\it i.e.\ }}
\newcommand{\etc}{{\it etc.\ }}

\newcommand{\be}{\begin{equation}}
\newcommand{\ee}{\end{equation}}
\newcommand{\bea}{\begin{eqnarray}}
\newcommand{\eea}{\end{eqnarray}}


% CROSS-REFERENCING
\def\Sref#1{Section~\ref{#1}\xspace}
\def\Fref#1{Figure~\ref{#1}\xspace}
\def\Tref#1{Table~\ref{#1}\xspace}
\def\Eref#1{Equation~\ref{#1}\xspace}
\def\Aref#1{Appendix~\ref{#1}\xspace}

\newcommand{\PaperOne}{Paper~I\xspace}
\newcommand{\PaperTwo}{Paper~II\xspace}

\newcommand{\StageOne}{Stage~1\xspace}
\newcommand{\StageTwo}{Stage~2\xspace}

% UNITS
\newcommand{\kms}{\ifmmode  \,\rm km\,s^{-1} \else $\,\rm km\,s^{-1}  $
\fi }
\newcommand{\kpc}{\ifmmode  {\rm kpc}  \else ${\rm  kpc}$ \fi  }
\newcommand{\pc}{\ifmmode  {\rm pc}  \else ${\rm pc}$ \fi  }
\newcommand{\Msun}{\ifmmode {\rm M_{\odot}} \else ${\rm M_{\odot}}$ \fi}
\newcommand{\Zsun}{\ifmmode {\rm Z_{\odot}} \else ${\rm Z_{\odot}}$ \fi}
\newcommand{\yr}{\ifmmode yr^{-1} \else $yr^{-1}$ \fi}
\newcommand{\hMsun}{\ifmmode h^{-1}\,\rm M_{\odot} \else $h^{-1}\,\rm
M_{\odot}$ \fi}

% COSMOLOGY
\newcommand{\LCDM}{$\Lambda{\rm CDM}$}
\newcommand{\MS}{Millennium Simulation\xspace}

% LENSING
\def\zd{z_{\rm d}}
\def\zs{z_{\rm s}}
\def\Dd{D_{\rm d}}
\def\Ds{D_{\rm s}}
\def\Dt{D_{\Delta t}}
\def\Dds{D_{\rm ds}}
\def\Sigmacrit{\Sigma_{\rm crit}}
\def\REin{R_{\rm Ein}}
\def\MEin{M_{\rm Ein}}

% SOFTWARE/HARDWARE
\def\sw{{\small\sc Space\,Warps}\xspace}
\def\SW{{\sc Space\,Warps}\xspace}
\def\Talk{{\small\sc Talk}\xspace}
\def\Letters{{\small\sc Letters}\xspace}
\def\Letter{{\small\sc Letter}\xspace}
\def\Dashboard{{\small\sc Dashboard}\xspace}
\def\cfhtls{{CFHTLS}\xspace}
\def\python{{\sc python}\xspace}
\def\gravlens{{\sc gravlens}\xspace}
\def\sextractor{{\sc SExtractor}\xspace}
\def\fitsjs{{\sc fitsjs}\xspace}
\def\humvi{{\sc HumVI}\xspace}
\def\af{{\sc ArcFinder}\xspace}
\def\AF{{\sc ArcFinder}\xspace}
\def\rf{{\sc RingFinder}\xspace}
\def\RF{{\sc RingFinder}\xspace}
\def\PH{{\sc Planet\,Hunters}\xspace}
\def\GZ{{\sc Galaxy\,Zoo}\xspace}

% TABLES:
\newcommand\nodata{ ~$\cdots$~ }%

% PROBABILITY THEORY
\def\pr{{\rm Pr}}
\def\data{{\mathbf{d}}}
\def\datap{{\mathbf{d}^{\rm p}}}
\def\training{{\mathbf{d}^{\rm t}}}
\def\trainingk{{\mathbf{d}^{\rm t}_k}}
\def\datai{d_i}
\def\datapi{d^{\rm p}_i}
\def\LENS{{\rm LENS}}
\def\saidLENS{{\rm ``LENS"}}
\def\NOT{{\rm NOT}}
\def\saidNOT{{\rm ``NOT"}}
\def\CM{\mathcal{M}}
\def\PL{\CM_{LL}}
\def\PD{\CM_{NN}}

% AGENT/SAMPLE BUREAUCRACY
\def\effort{N_{\rm C}}
\def\thiseffort{N_{{\rm C},k}}
\def\experience{N_{\rm T}}
\def\skill{{\langle \Delta I \rangle_{0.5}}}
\def\contribution{\skill^{\rm total}}
\def\information{\Delta I}
\def\Ns{J} % Number of subjects
\def\Nv{K} % Number of volunteers
\def\Ncands{N_{\rm det}} % Number of detected candidates

% COMMENTING
\usepackage[usenames]{color}
\newcommand{\question}[2]{\textcolor{red}{Question from #1: #2}}
\newcommand{\flag}[2]{\textcolor{red}{#1: #2}}
\newcommand{\todo}[2]{\textcolor{red}{\bf To Do: #1: #2}}
\newcommand{\new}[1]{\textcolor{blue}{#1}}

\def\oxford{Dept.\ of Physics, University of Oxford, Keble Road, Oxford, OX1 3RH, UK}
\def\kipac{Kavli Institute for Particle Astrophysics and Cosmology, Stanford University, 452 Lomita Mall, Stanford, CA 94035, USA}
\def\ipmu{Kavli IPMU (WPI), University of Tokyo, 5-1-5 Kashiwanoha, Kashiwa 277-8583, Japan}
\def\adler{Adler Planetarium, Chicago, IL, USA}
\def\lausanne{EPFL, Lausanne, Switzerland}
\def\zurich{Department of Physics, University of Zurich, Switzerland}

\def\pjmemail{\tt pjm@slac.stanford.edu}
\def\amemail{\tt anupreeta.more@ipmu.jp}


%%%%%%%%%%%%%%%%%%%%%%%%%%%%%%%%%%%%%%%%%%%%%%%%%%%%%%%%%%%%%%%%%%%%%%%%%%%%%%

\title[\sw]
{\SW: New Lens Candidates from the CFHT Legacy Survey}
    
\author[More et al.]{%
  % The \SW Collaboration includes:

% Principal Investigators (opt-out):
   \newauthor{%
    Anupreeta~More,$^{1}$\thanks{\amemail}
    Aprajita~Verma,$^{2}$
    Philip~J.~Marshall,$^{2,3}$
    Surhud~More,$^{1}$
    }
   \newauthor{%
    Elisabeth~Baeten,$^{4}$
    Julianne~Wilcox,$^{4}$
    Christine~Macmillan,$^{4}$
    Claude~Cornen,$^{4}$
   }
   \newauthor{%
    Amit~Kapadia,$^{5}$
    Michael~Parrish,$^{5}$
    Chris~Snyder,$^{5}$
    Christopher~P.~Davis,$^{3}$
    Raphael~Gavazzi,$^{6}$
    }
   \newauthor{%
    Chris~J.~Lintott,$^{2}$
    Robert~Simpson,$^{2}$
    David~Miller,$^{4}$
    Arfon~M.~Smith,$^{4}$
    Edward~Paget,$^{4}$
    }
   \newauthor{%
    Prasenjit~Saha,$^{7}$
    Rafael~Kueng,$^{7}$
    Thomas~E.~Collett,$^{8}$
    Matthias~Tecza,$^{2}$
    Michael~Baumer$^{3}$
    }
%
\medskip\\
$^1$\ipmu\\
$^2$\oxford\\
$^3$\kipac\\
$^4$\zooniverse\\
$^5$\adler\\
$^6$\paris\\
$^7$\zurich\\
$^8$\icg\\

}

%%%%%%%%%%%%%%%%%%%%%%%%%%%%%%%%%%%%%%%%%%%%%%%%%%%%%%%%%%%%%%%%%%%%%%%%%%%%%%

\begin{document}
             
\date{to be submitted to MNRAS}
\pagerange{\pageref{firstpage}--\pageref{lastpage}}\pubyear{2014}

\maketitle           

\label{firstpage}

%%%%%%%%%%%%%%%%%%%%%%%%%%%%%%%%%%%%%%%%%%%%%%%%%%%%%%%%%%%%%%%%%%%%%%%%%%%%%%

\begin{abstract} 

We present X new strong gravitational lens candidates found in the CFHT Legacy
survey by the \sw collaboration. 

\end{abstract}

% Full list of options at http://www.journals.uchicago.edu/ApJ/instruct.key.html

\begin{keywords}
  gravitational lensing   --
  methods: statistical    --
  methods: citizen science
\end{keywords}

\setcounter{footnote}{1}

%%%%%%%%%%%%%%%%%%%%%%%%%%%%%%%%%%%%%%%%%%%%%%%%%%%%%%%%%%%%%%%%%%%%%%%%%%%%%%

\section{Introduction}
\label{sec:intro}


In this paper, we ... try to answer the following questions:

\begin{itemize}

\item ...

\item ...

\item ...

\end{itemize}

This paper is organised as follows. In \Sref{sec:sw} we give a brief overview
of the \sw system, focusing on the aspects most relevant to the interpretation
of the results of this first lens search. In \Sref{sec:data} we introduce the
CFHTLS imaging data. ... We discuss the implications of our results for future
lens searches in \Sref{sec:discuss} and draw conclusions in
\Sref{sec:conclude}.


%%%%%%%%%%%%%%%%%%%%%%%%%%%%%%%%%%%%%%%%%%%%%%%%%%%%%%%%%%%%%%%%%%%%%%%%%%%%%%

\section{\sw Overview}
\label{sec:sw}

%%%%%%%%%%%%%%%%%%%%%%%%%%%%%%%%%%%%%%%%%%%%%%%%%%%%%%%%%%%%%%%%%%%%%%%%%%%%%%

\section{Data}
\label{sec:data}

Definitions: training subjects and test subjects. Sims and duds.

% % % % % % % % % % % % % % % % % % % % % % % % % % % % % % % % % % % % % % % 

\subsection{The CFHT Legacy Survey}
\label{sec:data:CFHTLS}

Describe survey. Refs. 

Why this one? Good IQ, deep, colorful, homogeneous. Precursor to Stage III and
IV imaging surveys, DES, KIDS, LSST etc. Already searched by robots: enables
comparison of techniques. Lenses not yet found by robots, detectable by
humans? 

Blind search strategy.
Preparation of data: divide survey into overlapping tiles. 


% % % % % % % % % % % % % % % % % % % % % % % % % % % % % % % % % % % % % % % 

\subsection{Image Presentation}
\label{sec:data:display}

Presentation of images. Uniform scales, to build intuition and avoid rescales
due to bright objects. Arcsinh stretch, to bring out low SB features. 
Approximately optimized, how? Examples of images.

% % % % % % % % % % % % % % % % % % % % % % % % % % % % % % % % % % % % % % % 

\subsection{Known Lenses}
\label{sec:data:knownlenses}


%%%%%%%%%%%%%%%%%%%%%%%%%%%%%%%%%%%%%%%%%%%%%%%%%%%%%%%%%%%%%%%%%%%%%%%%%%%%%%

\section{Results}
\label{sec:results}

Completeness and purity with regard to known lenses.

% % % % % % % % % % % % % % % % % % % % % % % % % % % % % % % % % % % % % % % 

\subsection{Stage 1 Classification}
\label{sec:results:stage1}


% % % % % % % % % % % % % % % % % % % % % % % % % % % % % % % % % % % % % % % 

\subsection{Stage 2 Classification}
\label{sec:results:stage2}


% % % % % % % % % % % % % % % % % % % % % % % % % % % % % % % % % % % % % % % 

\subsection{Comparison with ``Expert'' Classification}
\label{sec:results:experts}


% % % % % % % % % % % % % % % % % % % % % % % % % % % % % % % % % % % % % % % 

\subsection{New Lens Candidates}
\label{sec:results:newlenses}


%%%%%%%%%%%%%%%%%%%%%%%%%%%%%%%%%%%%%%%%%%%%%%%%%%%%%%%%%%%%%%%%%%%%%%%%%%%%%%

\section{Discussion}
\label{sec:discuss}

Differences with robots: what types of lenses are found at \sw?

Selection function. Missing system.

Further work.

%%%%%%%%%%%%%%%%%%%%%%%%%%%%%%%%%%%%%%%%%%%%%%%%%%%%%%%%%%%%%%%%%%%%%%%%%%%%%%

\section{Conclusions}
\label{sec:conclude}

Summary of CFHTLS project.

Crowd-sourced gravitational lens detection works, in terms of the
classification of the CFHTLS image dataset as described here, in the following
specific ways: 

\begin{itemize} 

\item Real (robotically-detected and expert-confirmed) lenses are
recovered/missed at similar/comparable/different rates to the
training set. 

\item Completeness, purity of \sw sample to known lenses.

\item Partial test of supervised vs unsupervised learning, comment.

\item Which real lenses were missed? False negatives, what we conclude from
that.

\item We found a
sample of new gravitational lens candidates with the following properties: ...

\item An expert-graded sample of size N, divided among several types, compares
as follows.

\item The SW lenses are different from the robotically (RingFinder and
ArcFinder) detected lenses, in the following ways.


\end{itemize}

Sum up, end.

%%%%%%%%%%%%%%%%%%%%%%%%%%%%%%%%%%%%%%%%%%%%%%%%%%%%%%%%%%%%%%%%%%%%%%%%
%%  ACKNOWLEDGMENTS
%%%%%%%%%%%%%%%%%%%%%%%%%%%%%%%%%%%%%%%%%%%%%%%%%%%%%%%%%%%%%%%%%%%%%%%%

\section*{Acknowledgements}
 
We thank all \Ncollaboration members of the \sw community for their
contributions to the project so far. A complete list of collaborators is
given at... In particular we would like to recognise the efforts of XXX,
YYY etc in moderating the discussion.

We are also grateful to Brooke Simmons, David Hogg, XXX and YYY for many useful
conversations about citizen science and gravitational lens detection. 
%
PJM was given support by the Royal Society, in the form of a research
fellowship, and by the U.S. Department of Energy under contract number DE-AC02-76SF00515.
%
AV acknowledges support from the Leverhulme Trust in the form of a research
fellowship.
%
The work of AM and SM was supported by World Premier International Research
Center Initiative (WPI Initiative), MEXT, Japan.
%
% Zooniverse / Sloan Foundation.
% 
% Other authors.
PJM and ES thank the Institute of Astronomy and Astrophysics, Academia Sinica
(ASIAA) and Taiwan's Ministry of Science and Technology (MOST) for their
financial support of the workshop ``Citizen Science in Astronomy'' in March
2014, at which some parts of the SWAP analysis was developed.


The \sw project is open source. 
The web app was developed at https://github.com/Zooniverse/Lens-Zoo
while the SWAP analysis software was developed at
https://github.com/drphilmarshall/SpaceWarps.


%%%%%%%%%%%%%%%%%%%%%%%%%%%%%%%%%%%%%%%%%%%%%%%%%%%%%%%%%%%%%%%%%%%%%%%%%%%%%%
%  APPENDICES
%%%%%%%%%%%%%%%%%%%%%%%%%%%%%%%%%%%%%%%%%%%%%%%%%%%%%%%%%%%%%%%%%%%%%%%%%%%%%%

% \appendix


%%%%%%%%%%%%%%%%%%%%%%%%%%%%%%%%%%%%%%%%%%%%%%%%%%%%%%%%%%%%%%%%%%%%%%%%%%%%%%
%  REFERENCES
%%%%%%%%%%%%%%%%%%%%%%%%%%%%%%%%%%%%%%%%%%%%%%%%%%%%%%%%%%%%%%%%%%%%%%%%%%%%%%

% MNRAS does not use bibtex, input .bbl file instead. 
% Generate this in the makefile using bubble script in scriptutils:

% bubble -f paper-lcr.tex references.bib 
% \input{paper-lcr.bbl}

\bibliographystyle{apj}
\bibliography{references}


%%%%%%%%%%%%%%%%%%%%%%%%%%%%%%%%%%%%%%%%%%%%%%%%%%%%%%%%%%%%%%%%%%%%%%%%%%%%%%

\label{lastpage}
\bsp

\end{document}

%%%%%%%%%%%%%%%%%%%%%%%%%%%%%%%%%%%%%%%%%%%%%%%%%%%%%%%%%%%%%%%%%%%%%%%%%%%%%%
