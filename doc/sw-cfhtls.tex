%%%%%%%%%%%%%%%%%%%%%%%%%%%%%%%%%%%%%%%%%%%%%%%%%%%%%%%%%%%%%%%%%%%%%%%%
%
% Space Warps II: CFHTLS 
%
%%%%%%%%%%%%%%%%%%%%%%%%%%%%%%%%%%%%%%%%%%%%%%%%%%%%%%%%%%%%%%%%%%%%%%%%

\documentclass[useAMS,usenatbib,a4paper]{mn2e}
%% letterpaper
%% a4paper

\voffset=-0.6in

% Packages:
\input psfig.sty
\usepackage{xspace}
\usepackage{graphicx}
\usepackage{amssymb}
\usepackage{amsmath}
\usepackage{longtable}

% Macros:
% JOURNALS
\newcommand{\apj}{ApJ}
\newcommand{\apjl}{ApJL}
\newcommand{\apjs}{ApJS}
\newcommand{\mnras}{MNRAS}
\newcommand{\apss}{Ap \& SS}
\newcommand{\aap}{A\&A}
\newcommand{\aj}{AJ}
\newcommand{\prd}{Phys. Rev. D}
\newcommand{\nat}{Nature}
\newcommand{\araa}{ARA\&A}
\newcommand{\jgr}{J. Geophys. Res.}
\newcommand{\pasp}{PASP}

% MISC
\newcommand{\etal}{et~al.~}
\newcommand{\eg}{{\it e.g.\ }}
\newcommand{\ie}{{\it i.e.\ }}
\newcommand{\etc}{{\it etc.\ }}

\newcommand{\be}{\begin{equation}}
\newcommand{\ee}{\end{equation}}
\newcommand{\bea}{\begin{eqnarray}}
\newcommand{\eea}{\end{eqnarray}}


% CROSS-REFERENCING
\def\Sref#1{Section~\ref{#1}\xspace}
\def\Fref#1{Figure~\ref{#1}\xspace}
\def\Tref#1{Table~\ref{#1}\xspace}
\def\Eref#1{Equation~\ref{#1}\xspace}
\def\Aref#1{Appendix~\ref{#1}\xspace}

\newcommand{\PaperOne}{Paper~I\xspace}
\newcommand{\PaperTwo}{Paper~II\xspace}

\newcommand{\StageOne}{Stage~1\xspace}
\newcommand{\StageTwo}{Stage~2\xspace}

% UNITS
\newcommand{\kms}{\ifmmode  \,\rm km\,s^{-1} \else $\,\rm km\,s^{-1}  $
\fi }
\newcommand{\kpc}{\ifmmode  {\rm kpc}  \else ${\rm  kpc}$ \fi  }
\newcommand{\pc}{\ifmmode  {\rm pc}  \else ${\rm pc}$ \fi  }
\newcommand{\Msun}{\ifmmode {\rm M_{\odot}} \else ${\rm M_{\odot}}$ \fi}
\newcommand{\Zsun}{\ifmmode {\rm Z_{\odot}} \else ${\rm Z_{\odot}}$ \fi}
\newcommand{\yr}{\ifmmode yr^{-1} \else $yr^{-1}$ \fi}
\newcommand{\hMsun}{\ifmmode h^{-1}\,\rm M_{\odot} \else $h^{-1}\,\rm
M_{\odot}$ \fi}

% COSMOLOGY
\newcommand{\LCDM}{$\Lambda{\rm CDM}$}
\newcommand{\MS}{Millennium Simulation\xspace}

% LENSING
\def\zd{z_{\rm d}}
\def\zs{z_{\rm s}}
\def\Dd{D_{\rm d}}
\def\Ds{D_{\rm s}}
\def\Dt{D_{\Delta t}}
\def\Dds{D_{\rm ds}}
\def\Sigmacrit{\Sigma_{\rm crit}}
\def\REin{R_{\rm Ein}}
\def\MEin{M_{\rm Ein}}

% SOFTWARE/HARDWARE
\def\sw{{\small\sc Space\,Warps}\xspace}
\def\SW{{\sc Space\,Warps}\xspace}
\def\Talk{{\small\sc Talk}\xspace}
\def\Letters{{\small\sc Letters}\xspace}
\def\Letter{{\small\sc Letter}\xspace}
\def\Dashboard{{\small\sc Dashboard}\xspace}
\def\cfhtls{{CFHTLS}\xspace}
\def\python{{\sc python}\xspace}
\def\gravlens{{\sc gravlens}\xspace}
\def\sextractor{{\sc SExtractor}\xspace}
\def\fitsjs{{\sc fitsjs}\xspace}
\def\humvi{{\sc HumVI}\xspace}
\def\af{{\sc ArcFinder}\xspace}
\def\AF{{\sc ArcFinder}\xspace}
\def\rf{{\sc RingFinder}\xspace}
\def\RF{{\sc RingFinder}\xspace}
\def\PH{{\sc Planet\,Hunters}\xspace}
\def\GZ{{\sc Galaxy\,Zoo}\xspace}

% TABLES:
\newcommand\nodata{ ~$\cdots$~ }%

% PROBABILITY THEORY
\def\pr{{\rm Pr}}
\def\data{{\mathbf{d}}}
\def\datap{{\mathbf{d}^{\rm p}}}
\def\training{{\mathbf{d}^{\rm t}}}
\def\trainingk{{\mathbf{d}^{\rm t}_k}}
\def\datai{d_i}
\def\datapi{d^{\rm p}_i}
\def\LENS{{\rm LENS}}
\def\saidLENS{{\rm ``LENS"}}
\def\NOT{{\rm NOT}}
\def\saidNOT{{\rm ``NOT"}}
\def\CM{\mathcal{M}}
\def\PL{\CM_{LL}}
\def\PD{\CM_{NN}}

% AGENT/SAMPLE BUREAUCRACY
\def\effort{N_{\rm C}}
\def\thiseffort{N_{{\rm C},k}}
\def\experience{N_{\rm T}}
\def\skill{{\langle \Delta I \rangle_{0.5}}}
\def\contribution{\skill^{\rm total}}
\def\information{\Delta I}
\def\Ns{J} % Number of subjects
\def\Nv{K} % Number of volunteers
\def\Ncands{N_{\rm det}} % Number of detected candidates

% COMMENTING
\usepackage[usenames]{color}
\newcommand{\question}[2]{\textcolor{red}{Question from #1: #2}}
\newcommand{\flag}[2]{\textcolor{red}{#1: #2}}
\newcommand{\todo}[2]{\textcolor{red}{\bf To Do: #1: #2}}
\newcommand{\new}[1]{\textcolor{blue}{#1}}

\def\oxford{Dept.\ of Physics, University of Oxford, Keble Road, Oxford, OX1 3RH, UK}
\def\kipac{Kavli Institute for Particle Astrophysics and Cosmology, Stanford University, 452 Lomita Mall, Stanford, CA 94035, USA}
\def\ipmu{Kavli IPMU (WPI), University of Tokyo, 5-1-5 Kashiwanoha, Kashiwa 277-8583, Japan}
\def\adler{Adler Planetarium, Chicago, IL, USA}
\def\lausanne{EPFL, Lausanne, Switzerland}
\def\zurich{Department of Physics, University of Zurich, Switzerland}

\def\pjmemail{\tt pjm@slac.stanford.edu}
\def\amemail{\tt anupreeta.more@ipmu.jp}



%%%%%%%%%%%%%%%%%%%%%%%%%%%%%%%%%%%%%%%%%%%%%%%%%%%%%%%%%%%%%%%%%%%%%%%%

\title[\sw II]
{\SW: II. Did we beat the lens finding robots in CFHTLS?}
    
\author[More et al.]{%
 % The \SW Collaboration includes:

% Principal Investigators (opt-out):
   \newauthor{%
    Anupreeta~More,$^{1}$\thanks{\amemail}
    Aprajita~Verma,$^{2}$
    Philip~J.~Marshall,$^{2,3}$
    Surhud~More,$^{1}$
    }
   \newauthor{%
    Elisabeth~Baeten,$^{4}$
    Julianne~Wilcox,$^{4}$
    Christine~Macmillan,$^{4}$
    Claude~Cornen,$^{4}$
   }
   \newauthor{%
    Amit~Kapadia,$^{5}$
    Michael~Parrish,$^{5}$
    Chris~Snyder,$^{5}$
    Christopher~P.~Davis,$^{3}$
    Raphael~Gavazzi,$^{6}$
    }
   \newauthor{%
    Chris~J.~Lintott,$^{2}$
    Robert~Simpson,$^{2}$
    David~Miller,$^{4}$
    Arfon~M.~Smith,$^{4}$
    Edward~Paget,$^{4}$
    }
   \newauthor{%
    Prasenjit~Saha,$^{7}$
    Rafael~Kueng,$^{7}$
    Thomas~E.~Collett,$^{8}$
    Matthias~Tecza,$^{2}$
    Michael~Baumer$^{3}$
    }
%
\medskip\\
$^1$\ipmu\\
$^2$\oxford\\
$^3$\kipac\\
$^4$\zooniverse\\
$^5$\adler\\
$^6$\paris\\
$^7$\zurich\\
$^8$\icg\\

}

%%%%%%%%%%%%%%%%%%%%%%%%%%%%%%%%%%%%%%%%%%%%%%%%%%%%%%%%%%%%%%%%%%%%%%%%

\begin{document}
             
\date{to be submitted to MNRAS}
\pagerange{\pageref{firstpage}--\pageref{lastpage}}\pubyear{2013}

\maketitle           

\label{firstpage}

%%%%%%%%%%%%%%%%%%%%%%%%%%%%%%%%%%%%%%%%%%%%%%%%%%%%%%%%%%%%%%%%%%%%%%%%

\begin{abstract} 

The CFHT Legacy Survey has been searched for strong lenses with
semi-automated algorithms both at galaxy and groups scales. With the aim
of improving these lens finding robots, we carry out a blind lens search
in the complete \cfhtls \textsc{wide} survey with \sw. We describe the training
sample used for both training the citizen scientists that participated
in \sw and calibrating their performance. We generate realistic looking
simulated samples of lenses both at galaxy and group-scales as part of
this training sample. We present 80 new strong gravitational lens
candidates discovered from the \sw-\cfhtls search out of which 40
candidates are promising. Furthermore, we compile a sample of false
positives which can be used for robot testing, first of its kind for the
lensing community. The simulated sample is also used to make a
comparison between human performance against the robots. We find that
the XX per cent of the known lens sample is recovered and the new lens
sample has XX completeness with respect to the simulated sample. We make
available the following data products: simulated sample, simulation code
and the false positives sample at XXX.

\end{abstract}

% Full list of options at http://www.journals.uchicago.edu/ApJ/instruct.key.html

\begin{keywords}
  gravitational lensing   --
  methods: statistical    --
  methods: citizen science
\end{keywords}

\setcounter{footnote}{1}

%%%%%%%%%%%%%%%%%%%%%%%%%%%%%%%%%%%%%%%%%%%%%%%%%%%%%%%%%%%%%%%%%%%%%%%%%%%%%%

\section{Introduction}
\label{sec:intro}

%{\it Describe various lens finding algorithms and the motivation for this particular project with \cfhtls. Already searched by robots: enables comparison of techniques. Lenses not yet found by robots, detectable by humans? }

The last few decades have seen a rise in the discoveries of strong
gravitational lenses owing to the plethora of interesting applications
they have in astrophysics and cosmology. Strong lenses are routinely
used to probe the dark matter distribution from galaxy (ref) to cluster
scales (ref), to  study distant young galaxies by using the lensing
magnification as a natural telescope (ref), to test the cosmological
model by constraining cosmological parameters such as the Hubble
constant (ref) and dark energy (ref) and many more. Even though strong
lenses are rare, since a foreground massive object needs to be
sufficiently aligned with a distant background source to produce
multiple images, systematic lens searches have led to discovery of over
500 lenses till date (XXX add mld url). 

Rarity of lenses implies searching for them is a painstaking task.
Efficient automated methods are thus imperative to finding a reasonably
complete and pure sample of strong lenses.   



Since the inception of the first citizen science project, Galaxy Zoo, to
classify galaxy morphologies \citep{LintottEtal2008}, several astronomy
and non-astronomy projects have been launched by the Zooniverse leading
to many interesting project. For example, PlanetHunters
has discovered XXX transiting exoplanets \citep{SchwambEtal2012},
Supernova XXX add ref and XXX add ref


In Paper I, we describe \sw, an online system that enables crowd-sourced
detection of gravitational lenses. In this paper (referred to as Paper
II), we describe our first lens search with \sw in the \cfhtls data.

This paper is organised as follows. In \Sref{sec:sw} we give a brief
overview of the \sw system, focusing on the aspects most relevant to the
interpretation of the results of this first lens search. In
\Sref{sec:data} we introduce the \cfhtls imaging data and the known lens
samples from the \cfhtls. In \Sref{sec:}, we explain the training sample
generated to aid the \sw users in the lens search. In
\Sref{sec:results}, we present the new lens candidates and our findings.
We discuss the implications of our results for future lens searches in
\Sref{sec:discuss} and draw conclusions in \Sref{sec:conclude}.

%%%%%%%%%%%%%%%%%%%%%%%%%%%%%%%%%%%%%%%%%%%%%%%%%%%%%%%%%%%%%%%%%%%%%%%%%%%%%%

\section{Data}
\label{sec:data}
\subsection{The CFHT Legacy Survey}
\label{sec:data:cfhtls}

The Canada-France-Hawaii Telescope Legacy Survey (CFHTLS) is a
photometric survey in five optical bands ($u^*g'r'i'z'$) carried out
with the wide-field imager MegaPrime which has a 1~deg$^2$ field-of-view
and a pixel size of 0.186\arcsec. The \cfhtls \textsc{WIDE} covers a total area
of 171~deg$^2$ on the sky and it consists of four fields W1, W2, W3 and
W4. The field W1 has the largest sky coverage of 63.65~deg$^2$. The
fields W2 and W4 have similar sky coverages of 20.32~deg$^2$ and
20.02~deg$^2$, respectively\footnote{These numbers are estimated from http://terapix.iap.fr/cplt/table\_syn\_T0006.html}.
The field W3 has a sky coverage of 42.87~deg$^2$ and is more than twice
as large as W2 and W4. 

The \cfhtls imaging is very homogeneous and has great image quality. Most of
the lensed arcs are much brighter in the $g$ band thus, deep imaging in
this band is desirable. The limiting magnitude is 25.47 for the $g$ band
which goes the deepest among all of the five bands. The mean seeing in
the $g$ band is 0.78\arcsec. The zero point to convert flux to AB
magnitude for all bands is 30. These characteristics make \cfhtls ideal
to do visual inspection for finding lenses.  We use the data from the
final T0007 release taken from the Terapix
website\footnote{{http://terapix.iap.fr/cplt/T0006-doc.pdf}}
for this work.

We note that the \cfhtls is a niche survey with a unique combination of
wide imaging with deep sensitivity. It is a precursor to the ongoing
wide imaging surveys such as the DES, KIDS and HSC and planned surveys
such as the LSST. Searching for lenses with \sw in the \cfhtls will
teach us important lessons and help prepare us for these larger imaging
surveys.



%%%%%%%%%%%%%%%%%%%%%%%%%%%%%%%%%%%%%%%%%%%%%%%%%%%%%%%%%%%%%%%%%%%%%%%%%%%%%%

\subsection{Existing \cfhtls lens samples}
\label{sec:data:kls}

The \cfhtls data has been searched for lenses using semi-automated algorithms,
primarily, in the $g$ band where the predominantly  faint blue  source
galaxies are bright compared to the predominantly red deflector galaxies.
Here, we briefly mention the lens samples which were known to the authors
prior to the lens search with \sw.

At galaxy-scales, we focus on two primary lens searches. The RingFinder
\citep{Gavazzi2014} was used for finding compact rings or arcs around centers
of isolated and massive early-type galaxies. By subtracting the PSF-matched
$i$-band images from the $g$-band images, the algorithm looks for excess flux
in the bluer $g$-band. An object detector measures the properties of these
residual blue features, and candidates meeting length-width ratio and
tangential alignment criteria are then visually inspected to form the final
sample. \citet{Gavazzi2014} first selected some 638,000 targets as either
photometrically-classified early type galaxies, or objects selected to have
red centers and blue outer parts, from the T06 CFHTLS data release catalogs. 
14370 were found to show detectable blue residuals, and 2524 were visually
inspected, having passed the automatic feature selection process. This led to
a sample of 42 good quality (\texttt{q\_flag} = 3) and 288 medium quality
(\texttt{q\_flag} = 2) lens candidates. In addition to this well defined
sample, \citet{Gavazzi2014} reported a further 71 serendipitously detected
lens candidates.  From this sample of ``RingFinder candidates,'' the SL2S team
found, during their follow-up campaign, 39 confirmed lenses (and 17 promising
candidates). We use this sample of 39 ``confirmed RingFinder lenses'' in our
completeness analysis.

The second galaxy-scale lens search was to find edge-on galaxy lenses in
the \cfhtls \citep{Sygnet2010} by selecting the galaxy's profile from the
output of {\sc Sextractor} in the $i$~band and with low inclination
angle. This sample has about 3 promising and a total of 18 lens
candidates.
 
On the other hand, the {\sc arcfinder} \citet{More2012} was used for
finding blue arc-like features in the complete \cfhtls data withouth any
pre-selection on the type of the lensing object. The search was carried
out in the g-band which is the most efficient wavelength to find typical
lensing galaxies. This sample, called the SARCS, has 55 promising and a
total of 127 lens candidates and consists of both galaxy and
groups/cluster scale lens candidates. Arc finding is better suited for
lensed images or arcs with larger image separations i.e. more massive
systems like groups and clusters. Thus, the SARCS sample has mostly
groups/cluster-scale lenses and a few galaxy-scale lenses.

For the purposes of transparency and to help a little with their training, 
the volunteers participating in \sw-\cfhtls lens search were made aware of
these known lens samples. Images containing the systems from the above samples
were labelled as ``Known Lens Candidates'' in the Talk forum, where volunteers
have the opportunity to discuss their findings with other volunteers and the
science team. 

XXX refer Paper I ?


%%%%%%%%%%%%%%%%%%%%%%%%%%%%%%%%%%%%%%%%%%%%%%%%%%%%%%%%%%%%%%%%%%%%%%%%%%%%%%

\subsection{Image Presentation}
\label{sec:data:impres}
{\it Preparation of data: divide survey into overlapping tiles. 

Presentation of images. Uniform scales, to build intuition and avoid rescales
due to bright objects. Arcsinh stretch, to bring out low SB features. 
Approximately optimized, how? Examples of images.}

We carry out a blind search over the \cfhtls fields where we, mainly, use
the g-r-i color information to look for signs of lensing. We extract
contiguous cutouts of size 81.84\arcsec\ (440 pixels). The neighbouring
cutouts have an overlapping region of 10\arcsec (54 pixels). If a
possible lens candidate is too close to the edge of a cutout, this
overlap allows the inspector to get a clearer view of the same candidate
in a neighbouring cutout. We note that since the images are shown
randomly, a given inspector may not necessarily come across the
neighbouring cutout unless the inspector classifies a lot of images.
However, this is not a problem since we our user base is extremely large
and we get multiple classifications of the same image.


%%%%%%%%%%%%%%%%%%%%%%%%%%%%%%%%%%%%%%%%%%%%%%%%%%%%%%%%%%%%%%%%%%%%%%%%%%%%%%

\section{Training sample: Simulated lenses}
\label{sec:ts}

{\it State the importance}

% % % % % % % % % % % % % % % % % % % % % % % % % % % % % % % % % % % % % % % 
\begin{figure*}
\begin{center}
\includegraphics[scale=1.0]{sw-cfhtls-figs/sim_cgq.pdf}
\caption{ \label{fig:sim}
Examples of the three types of simulated lenses.
}
\end{center}
\end{figure*}


\subsection{Methodology}
\label{sec:simmethod}

We create two main types of simulated lens sample a) galaxy-scale lenses and b)
group or cluster-scale lenses. The galaxy-scale lenses are further divided into two
types based on the nature of the background sources, namely, galaxies and
quasars. Below, we describe how each type of the lens sample was generated.


\subsubsection{Galaxy-scale lenses} 
\label{sect:gallens}

The $N_{\rm src}$ behind a lens is then calculated by doing the
following integral, 

\be
\label{eqn:nsrc}
N_{\rm src} = N_{\rm src}(>L_s,z_s) \int_{z_l}^\infty \sigma_{\rm lens}(\sigma,z_l,z_s,q) D_s^2 (1+z_s)^2 \frac{\rm d\chi}{{\rm d}z_s}
{\rm d}z_s 
\ee

\be
\label{eqn:nlum}
 N_{\rm src}(>L_s,z_s)= \int_{L_{min}}^\infty \Phi(L_s,z_s) {\rm d}L_s
\ee

where $\Phi(L_s,z_s)$ is the source luminosity function per unit comoving
volume, $q$ is the projected axis ratio of the lens ellipticity, $\chi$ and
$D_s$ are the comoving and angular diameter distances to the source,

We use the elliptical galaxy (LRG) catalog from the \cfhtls XXX to select
all the foreground galaxies (e.g. $z<1$) that are potential lenses for the
simulated sample. We exclude all those galaxies whose positions match with the
lensing galaxies from the known \cfhtls~SL2S lens samples \cite{More2012}
within 2~arcsec (XXX check). 

First, we calculate the luminosity and velocity dispersion of each potential
lensing galaxy using the CFHT Megacam $g$ and $r$ band magnitudes along with the
photometric redshift ($z_l$) from the LRG catalog.  The Megacam magnitudes are
converted to SDSS magnitudes\footnote{
    http://www3.cadc-ccda.hia-iha.nrc-cnrc.gc.ca/megapipe/docs/filters.html} and
are further k-corrected to redshift $z=0.1$ \citep{Frei1994}. We assume that the
evolution of galaxy luminosities is similar to that determined by
\citep{Faber2007}, that is, a decline of $1.5$ in the $m_{r*}$ from redshift $z=1$
to $z=0$ (see Eq.~\ref{magstar}). 

\be
\frac{L}{L_{*}}=10.0^{-0.4~(m_{r \rm SDSS}-m_{r*})} 
\ee

where $m_{r*}$ is
\be
\label{magstar}
m_{r*}=-20.44+ 1.5~(z_{l}-0.1) \,.
\ee

We use the $L-\sigma$ relation from
\citep{Parker2007} to get the velocity dispersion as given in Eq.~\ref{magstar2}.

\be
\label{magstar2}
\sigma=142 \left(\frac{L}{L_{*}}\right)^{1/3} 
\ee

Next, in order to decide whether a galaxy is likely to act as a strong lens, we
calculate the lens cross-section ($\sigma_{\rm lens}$) and the number of sources
($N_{\rm src}$) that are in the background. Following \citep{Keeton2000a}, the lens
cross-section is calculated analytically for an isothermal model and is given by 

\be
\sigma_{\rm lens}=b_I^2 \, \int_0^{2\pi} 0.5 r^2(\theta) d\theta
\ee

where $b_I$ is 
\be
b_I = b_{\rm SIS} \epsilon_3/sin^{-1}(\epsilon_3) \,,
\ee

the eccentricity ($\epsilon_3$) is 
\be
\epsilon_3=(1-q_3^2)^{1/2}
\ee

and the projected axis ratio is given by
\be
q_k=\sqrt{q_3^2 {\rm sin}^2i_e+{\rm cos}^2i_e} \,.
\ee

In the above equations, $q_3$ is the 3d axis ratio of the ellipsoid and $i_e$ is
the inclination angle. Also, $b_{\rm SIS}= 4\pi
\frac{c^2}{\sigma^2}\frac{D_{s}}{D_{ls}D_l}$ and is referred to as the
Einstein radius where $D_s$,$D_l$ and $D_{ls}$ are angular diameter distances to
the source, the lens and between the lens and source, respectively.

%Comparing \citep{keeton00a} with \citep{kormann94} suggests that $b_I = sqrt(q)*b_{SIS}$ 

Next, if the foreground galaxy can act as a lens and has at least one source in the
background, then we determine a redshift ($z_s$) and $i$-band magnitude of the
background source(s). We assume two types of background sources namely, galaxies
and quasars. For each source, the redshift and magnitude are generated by drawing randomly from
the following redshift and luminosity distributions. For
galaxies, we assume the redshift distribution is 

\be
\label{eqn:ps}
p_s=\frac{\beta z_s^2 {\rm exp}({\frac{z_s}{z_0(m_{\rm lim})}})^\beta}{\Gamma(3/\beta)z_0^3(m_{\rm lim})}
\ee

where $\beta=3/2$ and $z_0(m_{\rm lim})=0.13m_{\rm lim} - 2.2$ and the
luminosity function is

\be
\label{eqn:ns}
n_s=\int^{m_{\rm lim}}_{-\infty} \frac{n_0 {\rm d}m}{\sqrt{10^{2a(m_1-m)}+10^{2b(m_1-m)}}}
\ee

with parameters $a=0.30$, $b=0.56$, $m_1=20$ and $n_0=3\times10^3~deg^{-2}$ as
given in \citep{Faure2009} and references therein. For quasars, we calculate
the luminosity function by following the prescription of \citep{Ogur2010} and
use k-corrections by \citep{Richards2006}. 

%alp=-0.5;
%kcorr=-2.5*(1 + alp)*log10(1+zz);
%Dlum=cc.Dlofz(zz)/p.hval;
%DM=5*log10(Dlum);
%Mabs=mag-kcorr-DM-25.0;
The luminosity function is expressed as
\be
\frac{{\rm d}\Phi}{{\rm d}M}=\frac{\Phi_{*}}{10^{0.4(\alpha+1)(M_{\rm abs}-M_{*})} + 10^{0.4(\beta+1)(M_{\rm abs}-M_{*})} }
\ee

where the normalization, $\phi_{*}=5.34\times10^{-6} h^3$ Mpc$^{-3}$ and break
magnitude, $M_*=-20.90 + 5 {\rm log} h - 2.5 {\rm log} f(z)$. The redshift
dependent factor in $M_*$ is given by

\be
f(z)=\frac{e^{\zeta z_s}(1+e^{\xi z_*})}{(\sqrt{e^{\xi z_s}}+\sqrt{e^{\xi z_*}})^2} \,.
\ee
We adopt the best-fit values $\zeta=2.98$, $\xi=4.05$, $z_{*}=1.60$
\citep{Oguri2010}. For the faint end slope, we use $\beta=-1.45$ whereas for
the bright end slope, we use $\alpha=-3.31$ when $z_s<3$ and $\alpha=-2.58$ at
higher redshifts, as prescribed by \citep{Oguri2010}. 
respectively. We note that when calculating $N_{\rm src}$, the source
number density is artificially boosted by a factor (see Table~\ref{tab:thresh})
to increase the occurrence of simulated lenses. This helps in creating a large
enough sample to carry out various performance tests.

Next, we determine properties of the background source for every lens. We follow
similar procedures for both background galaxies and quasars. For simplicity, we
simulate a single background source behind every lens. In order to select one
background source from the $N_{\rm src}$ per lens, we do ray-tracing for all of the
$N_{\rm src}$ sources with {\sc gravlens} \citep{Keeton2000} and choose sources that
satisfy criteria as given below. We determine fluxes of the lensed images
and the total magnification of each of the lensed source. We draw a random
source for which the flux of the second brightest lensed image and the total
magnification of all lensed images are above the thresholds given in
Table~\ref{tab:thresh}.

Since we want to produce realistic looking lens systems, we simulate lenses in
each of the five \cfhtls~filters. The colors of the background galaxies are drawn
randomly from the photometric CFHTLenS catalog
\citep{Hildebrandt2012,Erben2013}.  Similarly, we use a quasar catalog from the
SDSS Data Release 9 \citep{Paris2012} from which colors are drawn to simulate
quasar lenses. Next, we assume deVaucoleur's profile to account for the size
and shape of the galaxies. The ellipticity and the position angle are drawn
randomly between the range given in Table~\ref{tab:thresh}. The effective
radius of the galaxy is estimated from the Luminosity$-$size relation
\citep{Bernardi2003} given by 
\be
R_{\rm eff}= 10^{0.52} \frac{L_r^{2/3}}{{(1+z_s)}^2}
\ee
where $L_r=L_s/10^{10.2}$. On the other hand, quasars are assumed to follow a
Gaussian profile where the $\sigma$ is equated to that of the median seeing for
every filter. The median seeing values are taken from Table 4 of the official
Terapix T0007 release explanatory document \footnote{
    http://terapix.iap.fr/cplt/T0007/doc/T0007-doc.pdf}.  

Once all the parameters are determined for the lens and source models, {\sc
gravlens} is used to simulate lensed images.  After accounting for the shot
noise in the lensed images and convolving them with the median seeing in each of
the filters, the simulated image is added to the real \cfhtls~image centered
on the lensing galaxy. Note that we ensure that the lensed galaxies and
lensed quasars do not have the same lensing galaxy in the foreground. Similarly,
the lensing galaxies from the galaxy-scale lenses are distinct from the central
galaxies of groups-scale lenses which are described in the following section.


\subsubsection{Groups-scale lenses} 

At group or cluster-scales, the brightest group galaxy (BGG) at the center alone
does not cause strong lensing. We need to account for the extra convergence
arising from the dark matter component as well as satellite galaxies, at least,
in the inner regions which are typically responsible for the multiple lensed
images \citep{Oguri2005,Oguri2006}. Owing to the lack of an appropriate group
catalog for our purposes, we create a basic group catalog based on the
magnitudes and photometric redshifts available for the \cfhtls. We select all
galaxies with 10$^{10.8} M_\odot$ as the BGGs. We select the member galaxies
such that their photometric redshifts are within $\delta z = 0.01$ of the BGG
and within an aperture of $250$~Kpc. 

We adopt an isothermal ellipsoid for the BGG and members whenever the
ellipticities are available else we use an isothermal sphere. On the other hand, we
adopt an NFW profile for the underlying dark matter halo. Assuming a constant
mass-to-light ratio of $3 \times 0.7$~h~$M_{*}/L_{*}$, we use the BGG luminosity
to estimate the stellar mass. The stellar mass$-$halo mass relation
\citep{behroozi13}, including random scatter, is then used to calculate the halo
mass for the lens. Given the halo mass, other key parameters such as the scale
radius ($r_s$) and the density at the scale radius ($\rho_s$) can be determined for an
NFW profile. 

As described in Section~\ref{sect:gallens}, we calculate the luminosity and
velocity dispersion for the BGG and each of the member galaxies. Next, we
calculate the lens cross-section for each potential lensing group. The
complexity in the lens models makes it analytically intractable to calculate the
size of the caustics\footnote{The lens mass distribution determines size and
shape of the caustics. Any source located within the caustics will form multiple
lensed images which is the criteria for strong lensing. To further
understand caustics, see XXX.}.  Hence, we use {\sc
gravlens} to determine the area covered by the caustics. We consider only
galaxies as our background source population since group or cluster-scale quasar
lenses are not expected to be found in the \cfhtls (check XXX).  Following the same
procedure as described in Section~\ref{sect:gallens}, we calculate the number of
galaxies expected to lie behind every potential lensing group (see
Eq.~\ref{eqn:nsrc}). As before, for each background galaxy within the lens cross-section, a
redshift and an $i$-band magnitude is determined by drawing galaxies randomly
from the respective distributions (see Eqs.~\ref{eqn:ps}-\ref{eqn:ns}). 

All those groups that are found to have no background galaxies within the
cross-sectional area are rejected and the rest are included as potential lenses.
As mentioned earlier, we artificially boost the total number of sources behind
every lens but ensure that (check XXX) the statistical properties such as the profile of the
image separation distribution are not affected (see \Fref{fig:remudist}). We
follow the same procedure and apply the same thresholds to determine properties
of the lensed galaxies for every lens as are described for galaxy-galaxy lenses
in the previous section. The simulated images are added to the real \cfhtls~images
with the BGGs as the center.

% % % % % % % % % % % % % % % % % % % % % % % % % % % % % % % % % % % % % % % 

\subsection{Simulated Lens Sample and Catalog Description}

\begin{table}
\begin{center}
\caption{ \label{tab:thresh} 
Thresholds used in the selection of the simulated lenses. }
\begin{tabular}{l l l l l}
\hline
Name  &  \multicolumn{2}{c} {gal}  & \multicolumn{2}{c}{qso} \\ 
      & min  &  max  & min & max \\
\hline
\hline
Source Redshift  & 1.0 & 4.0  & 1.0  & 5.9 \\
Source Magnitude & 21.0 & 25.5 & 21.0 & 25.5 \\

boost factor & 100 (40$\dagger$)  &  & 1200 & \\

Second brightest image & 23  & & 23 & \\
Total magnification & 19 & & 20 & \\

Lens shear strength &  0.001 & 0.02 &  0.001 & 0.02 \\
Lens shear pa &  0 & 180 & 0 & 180  \\
Source ellipticity & 0.1 & 0.6 & & \\
Source PA & 0 & 180 & & \\
\hline
\end{tabular}
{ $\dagger$} -- corresponds to the factor used for Groups scale lenses. 
\end{center}
\end{table}

\begin{figure}
\begin{center}
\includegraphics[scale=1.0]{sw-cfhtls-figs/distrib_remu.pdf}
\caption{ \label{fig:remudist}
Einstein radius distribution for all types of lenses. The dashed-dotted (blue)
curves show the theoretical prediction assuming an SIS model at galaxy-scales
and a total (NFW+Hernquist) model at groups-scales taken from \citep{More2012}.
}
\end{center}
\end{figure}

In this section, we describe some of the properties of our simulated sample for
each of the three types of lens samples.

The \Fref{fig:remudist} shows Einstein radius distribution for the
galaxy-scale (dashed for background quasars and dotted for background galaxies)
and groups-scale simulated lenses. For comparison, we show the expected
distributions (blue dashed-dotted curves) for an SIS-like density profile at
galaxy-scales and an NFW+Hernquist profile at groups-scales. The 
theoretical curves are taken from \citep{More2012} wherein the models are
explained in detail. We note that the model we adopt at groups-scale also
includes SIS or SIE components for the group members unlike the theoretical
prediction. The theoretical curves have arbitrary normalizations.

We show the redshift and magnitude distributions of the lensing galaxies in the
left and right panels of \Fref{fig:lensprop}
respectively. Furthermore, we overplot the distributions of respective
properties of the SARCS lenses from \citep{More2012} for comparison with arbitrary
normalizations. We note qualitative similarities between the simulated and the
real lens samples.

Similarly, we show the ellipticity and the position angle of the simulated lens
galaxy population extracted from the T0007 release of the \cfhtls~catalogs in the
left and right panels of \Fref{fig:lensprop}, respectively. As before, the
dashed-dotted (blue) curves show the same distribution for the SARCS lens
population with arbitrary normalization \citep{More2012}.

%We show the source redshift and magnitude distributions for each of the three
%lens samples in the left and right panels of \Fref{fig:szmdist},
%respectively. The peak of the source redshift distribution both for the galaxies
%and quasars is known to be between $2<z<3$ from various lens samples (add
%references XXX). This consistent with our simulated sample as expected. The
%magnitude distribution of the simulated sample, on the other hand, is
%specifically tailored to the requirements of \sw.  The goal was to generate a
%lens sample that has a fair balance of both bright and faint lensed images since
%the training sample should neither be too easy nor too difficult for the citizen
%scientists.  

We produce catalogs with lens and source properties for each of the three types
of lenses. These catalogs are available XXX. The catalogs typically have lens
position, redshift, magnitudes, Einstein radius, ellipticity (whenever
available) and shear (for galaxy-scale lenses only). For the background
sources, we provide the offset from the lens center, redshift, magnitudes,
total magnification, number of lensed images. Additionally, ellipticity and
effective radius when the background sources are galaxies.

%magdiff=5*0.1;
%zdiffq=0.1;

\begin{figure*}
\begin{center}
\includegraphics[scale=1.3]{sw-cfhtls-figs/lensprop.pdf}
\caption{ \label{fig:lensprop}
Distributions of properties of the lensing galaxies of the simulated
sample compared to the known lens sample SARCS XXXX check? 
}
\end{center}
\end{figure*}


%
%\begin{figure}
%\begin{center}
%\includegraphics[scale=0.95]{sw-cfhtls-figs/distrib_reff.eps}
%\caption{ \label{fig:reffdist}
%Distributions of magnification (left) and source effective radius in units of
%pixel (right).
%}
%\end{center}
%\end{figure}


% % % % % % % % % % % % % % % % % % % % % % % % % % % % % % % % % % % % % % % 

\section{Training sample: Duds and False positives}
\label{sec:dfp}

A good training sample consists of a representative set of objects that
one wants to find and another set of objects which appear to be from the
former set but are of different origin in reality and which one can
learn to discard efficiently. Indeed, we wanted to have a good training
sample for the \sw users so that they can correctly identify the true
lens candidates. Hence, in addition to the simulated lenses, we
added a sample of duds and false positives to the training sample. Duds
are images which have been visually inspected by experts and confirmed
to contain no lenses.  False positives are systems which look like
lenses but are not, for example, spiral galaxies, starforming galaxies,
chance alignments of features arranged in a lensing configuration and
stars. 

We selected a sample of 450 duds for the Stage I classification in \sw
(see XXX) and a sample of XXX false positives for the Stage II
inspection (see XXX). The sample of false positives was selected from
the candidates which passed the Stage I of \sw. We note that this is the
first time, we have a systematically compiled sample of visually
inspected false positives by the \sw users and categorized by the
science team. Such a sample is tremendously helpful for training and
understanding performances of various lens finding algorithms (refer
james, sherry's paper ? XXX)
{\bf TBA data products; make the FP sample available}

%%%%%%%%%%%%%%%%%%%%%%%%%%%%%%%%%%%%%%%%%%%%%%%%%%%%%%%%%%%%%%%%%%%%%%%%%%%%%%

\section{Results}
\label{sec:results}
% % % % % % % % % % % % % % % % % % % % % % % % % % % % % % % % % % % % % % % 

\subsection{\sw Lens Sample}

\subsubsection{Selection of \sw lens candidates}
\label{sec:results:stage1}

\begin{figure}
\begin{center}
\includegraphics[scale=1.0]{sw-cfhtls-figs/poffl_expr_ncl.pdf}
\caption{ \label{fig:comp_exp} Comparison of the P(lens) with the expert
grades and number of classification for each subject.  }
\end{center}
\end{figure}


% % % % % % % % % % % % % % % % % % % % % % % % % % % % % % % % % % % % % % % 

\subsection{New lens candidates from \sw}
\label{sec:results:disc}

{\bf \textcolor{red} {CHANGE THE TABLE, CHECK IF ALL FIGS ARE UPDATED,
ADD ISD PLOT, CHECK FOR STAGE 1 -- COMPLETENESS OF SIMS AND KNOWN LENSES
and P value comparison, what about the duplicate lenses near the borders
- stats on those?, MAKE A STATS TABLE that gives various numbers and
fractions of (new and known) lens candidates detected at stages 1 and 2.}}

{\it Describe the newly found lenses and explain for a few
cases why they were missed }

\begin{table}
\begin{center}
\caption{ \label{tab:stats} 
Statistics of detections in \sw }
\begin{tabular}{l l l l l l}
\hline
   &  \multicolumn{2}{c} {Stage I}  & \multicolumn{3}{c}{Stage II} \\ 
      & KnCa  &  KnCo  & KnCa & KnCo & NeCa \\
\hline
\hline
Number  & 1.0 & 1.0 & 1.0  & 1.0  & 1.0 \\
Fraction  & 1.0 & 1.0 & 1.0 & 1.0 & 1.0 \\
P(lens) & 0.95 & 0.95 & 0.3 & 0.3 & 0.3 \\

\hline
\end{tabular}
\end{center}
{KnCa}-- Known lens candidates \\
{KnCo}-- Known confirmed lenses \\
{NeCa}-- New lens candidates  \\
\end{table}



\onecolumn

\begin{center}
\begin{longtable}{lcccccccccr}
\caption{ \label{tab:swcands} 
Thresholds used in the selection of the simulated lenses. }\\
\hline
SW ID & Name & RA & Dec &  $z_{\rm phot}$ & $m_i$ & $R_{\rm A}$ & R$_{avg}$ & ZooID & P & Comments  \\
  &  & (deg) & (deg) &  & (mag) &  (") &  &  & & \\ 
\hline
\endfirsthead
\hline
SW ID & Name & RA & Dec &  $z_{\rm phot}$ & $m_i$ & $R_{\rm A}$ & R$_{avg}$ & ZooID & P & Comments  \\
  &  & (deg) & (deg) &  & (mag) &  (") &  &  & & \\ 
\hline
\endhead
\hline
\multicolumn{11}{p{18cm}}{
The column Comments has two type of notes. The first is about the lens
image configuration where the symbols mean the following A: Arc, D: Double, Q:
Quad, R: Ring. The second is a comment on the type of lens assessed
visually. Note that this classification is not based on colors or spectral
analysis. The symbols are E: Elliptical, S: (face on) Spiral, G: Group-scale, D:
Edge on disk, R: Red starforming galaxy.  This galaxy falls within the masked region as per the catalog from
which the magnitudes and the redshift are extracted.  
}\\  
\endlastfoot
 SW1 & CFHTLS J020338-051901 &   30.9097 &   -5.3171 &  0.4 & 18.8 &  1.6 &  2.0 & ASW0001sqw &  0.7  &  D,E   \\ 
 SW2 & CFHTLS J020341-074722 &   30.9223 &   -7.7897 &  0.5 & 18.8 &  1.6 &  2.0 & ASW000993q &  1.0  &  A,E   \\ 
 SW3 & CFHTLS J020457-110309 &   31.2392 &  -11.0526 &  0.7 & 20.0 &  1.6 &  2.3 & ASW00099b9 &  1.0  &  A,R   \\ 
 SW4 & CFHTLS J020642-095157 &   31.6751 &   -9.8659 &  --  &  --  &  0.9 &  2.0 & ASW0001ld7 &  1.0  &  A,R   \\ 
 SW5 & CFHTLS J020810-040220 &   32.0450 &   -4.0389 &  --  &  --  &  1.8 &  1.3 & ASW0001c3j &  1.0  &  A,R   \\ 
 SW6 & CFHTLS J020832-043315 &   32.1339 &   -4.5544 &  --  &  --  &  1.6 &  2.3 & ASW0002asp &  1.0  &  A,R   \\ 
 SW7 & CFHTLS J020848-042427 &   32.2011 &   -4.4076 &  --  &  --  &  1.1 &  2.7 & ASW0002bmc &  1.0  &  D,D   \\ 
 SW8 & CFHTLS J020849-050429 &   32.2078 &   -5.0748 &  --  &  --  &  0.9 &  2.3 & ASW0002qtn &  1.0  &  A,R   \\ 
 SW9 & CFHTLS J021021-093415 &   32.5898 &   -9.5711 &  0.4 & 18.4 &  2.7 &  1.3 & ASW0002k40 &  0.5  &  D,S   \\ 
SW10 & CFHTLS J021057-084450 &   32.7415 &   -8.7474 &  --  &  --  &  2.5 &  1.7 & ASW0002p8y &  0.4  &  A,G   \\ 
SW11 & CFHTLS J021221-105251 &   33.0881 &  -10.8811 &  0.3 & 17.9 &  1.8 &  2.0 & ASW0002dx7 &  1.0  &  D,E/S   \\ 
SW12 & CFHTLS J021225-085211 &   33.1051 &   -8.8697 &  0.8 & 19.5 &  2.1 &  1.7 & ASW00024id &  1.0  &  R,R   \\ 
SW13 & CFHTLS J021230-074727 &   33.1264 &   -7.7909 &  0.9 & 20.3 &  1.7 &  1.7 & ASW0001ze0 &  0.5  &  A,R/G   \\ 
SW14 & CFHTLS J021317-084819 &   33.3234 &   -8.8055 &  0.5 & 19.8 &  1.3 &  1.3 & ASW00024q6 &  0.6  &  A,R/E   \\ 
SW15 & CFHTLS J021514-092440 &   33.8109 &   -9.4111 &  0.7 & 19.9 &  2.6 &  1.3 & ASW00021r0 &  0.5  &  A,R/G   \\ 
SW16 & CFHTLS J022016-102446 &   35.0688 &  -10.4129 &  0.6 & 19.4 &  1.5 &  2.3 & ASW0009a4w &  0.7  &  D,E   \\ 
SW17 & CFHTLS J022359-083651 &   35.9995 &   -8.6144 &  --  &  --  &  3.1 &  1.3 & ASW0004iye &  0.7  &  A,E   \\ 
SW18 & CFHTLS J022406-062846 &   36.0256 &   -6.4796 &  0.4 & 19.6 &  0.9 &  2.3 & ASW0003wsu &  1.0  &  A,E   \\ 
SW19 & CFHTLS J022409-105808 &   36.0398 &  -10.9689 &  --  &  --  &  4.8 &  3.0 & ASW0004dv8 &  1.0  &  A,G   \\ 
SW20 & CFHTLS J022533-053204 &   36.3888 &   -5.5346 &  0.5 & 19.4 &  3.6 &  2.0 & ASW0004m3x &  0.7  &  A,R/G   \\ 
SW21 & CFHTLS J022716-105602 &   36.8186 &  -10.9341 &  0.4 & 17.3 &  1.8 &  2.0 & ASW0009ab8 &  0.8  &  A,E/G   \\ 
SW22 & CFHTLS J022745-062518 &   36.9387 &   -6.4218 &  0.6 & 20.5 &  1.2 &  1.7 & ASW0003s0m &  0.7  &  A,R   \\ 
SW23 & CFHTLS J022805-051733 &   37.0236 &   -5.2927 &  0.4 & 18.8 &  1.4 &  2.3 & ASW0009ans &  1.0  &  Q,E   \\ 
SW24 & CFHTLS J022817-080242 &   37.0727 &   -8.0452 &  0.5 & 19.6 &  0.9 &  2.0 & ASW00042d9 &  1.0  &  A,E/R   \\ 
SW25 & CFHTLS J022843-063316 &   37.1794 &   -6.5547 &  0.5 & 19.1 &  1.8 &  1.3 & ASW0003r6c &  0.4  &  D/A,E   \\ 
SW26 & CFHTLS J023008-054038 &   37.5359 &   -5.6774 &  0.6 & 19.7 &  1.9 &  2.3 & ASW0003r61 &  0.6  &  A,E   \\ 
SW27 & CFHTLS J023010-110409 &   37.5420 &  -11.0694 &  0.9 & 20.6 &  1.2 &  1.7 & ASW0004fgb &  0.7  &  A,R   \\ 
SW28 & CFHTLS J023051-082423 &   37.7141 &   -8.4064 &  --  &  --  &  0.8 &  2.3 & ASW000412m &  0.6  &  A,E   \\ 
SW29 & CFHTLS J023123-082535 &   37.8468 &   -8.4266 &  --  &  --  &  1.2 &  2.3 & ASW0004xjk &  0.4  &  A,R   \\ 
SW30 & CFHTLS J023315-042243 &   38.3133 &   -4.3789 &  0.7 & 19.7 &  1.0 &  2.0 & ASW00050sk &  1.0  &  A,R   \\ 
SW31 & CFHTLS J023325-053104 &   38.3547 &   -5.5178 &  0.5 & 18.8 &  1.3 &  1.7 & ASW0005ire &  1.0  &  Q,M   \\ 
SW32 & CFHTLS J023453-093032 &   38.7232 &   -9.5089 &  0.5 & 19.8 &  0.7 &  1.7 & ASW00051ld &  0.5  &  A,D   \\ 
SW33 & CFHTLS J084833-044051 &  132.1385 &   -4.6809 &  0.7 & 20.2 &  0.9 &  1.7 & ASW0004wgd &  0.9  &  A,R   \\ 
SW34 & CFHTLS J084841-045237 &  132.1708 &   -4.8772 &  0.3 & 19.0 &  1.0 &  2.3 & ASW0004nan &  1.0  &  A,E   \\ 
SW35 & CFHTLS J084941-051650 &  132.4216 &   -5.2808 &  0.4 & 19.1 &  1.5 &  1.3 & ASW0004nh3 &  0.6  &  A,E/S   \\ 
SW36 & CFHTLS J085135-052232 &  132.8980 &   -5.3756 &  --  &  --  &  1.9 &  1.3 & ASW0004n2x &  0.7  &  D,R   \\ 
SW37 & CFHTLS J085317-020312 &  133.3233 &   -2.0535 &  0.7 & 20.6 &  1.2 &  1.3 & ASW0000vtc &  0.7  &  A,R   \\ 
SW38 & CFHTLS J090218-053924 &  135.5790 &   -5.6567 &  --  &  --  &  2.0 &  1.3 & ASW0000g95 &  1.0  &  A,R/E   \\ 
SW39 & CFHTLS J090248-010232 &  135.7020 &   -1.0424 &  0.4 & 19.1 &  1.4 &  1.7 & ASW000096t &  0.8  &  D,E   \\ 
SW40 & CFHTLS J090308-043252 &  135.7840 &   -4.5479 &  --  &  --  &  1.2 &  2.0 & ASW00007mq &  0.8  &  A,E   \\ 
SW41 & CFHTLS J090319-040146 &  135.8311 &   -4.0297 &  --  & 19.8 &  1.2 &  1.7 & ASW00007ls &  0.6  &  A,R/E   \\ 
SW42 & CFHTLS J090333-005829 &  135.8890 &   -0.9749 &  --  &  --  &  2.1 &  1.7 & ASW00008a0 &  1.0  &  A/D,E/G   \\ 
SW43 & CFHTLS J135724+561614 &  209.3540 &   56.2707 &  --  &  --  &  2.6 &  1.3 & ASW0006e0o &  1.0  &  D,E   \\ 
SW44 & CFHTLS J135755+571722 &  209.4827 &   57.2897 &  0.8 & 20.2 &  1.3 &  2.0 & ASW0005ma2 &  1.0  &  D,D   \\ 
SW45 & CFHTLS J140027+541028 &  210.1160 &   54.1745 &  --  &  --  &  1.2 &  1.3 & ASW0006a07 &  0.8  &  Q,R/E   \\ 
SW46 & CFHTLS J140030+574437 &  210.1260 &   57.7437 &  0.4 & 18.2 &  2.0 &  2.3 & ASW0009bp2 &  0.9  &  A,E   \\ 
SW47 & CFHTLS J140425+520506 &  211.1062 &   52.0850 &  0.4 & 18.9 &  1.4 &  1.3 & ASW0005o0w &  0.9  &  D,E   \\ 
SW48 & CFHTLS J140522+574333 &  211.3426 &   57.7259 &  0.7 & 19.7 &  1.0 &  2.7 & ASW000619d &  1.0  &  A,R   \\ 
SW49 & CFHTLS J140622+520942 &  211.5958 &   52.1617 &  0.7 & 20.3 &  1.2 &  2.3 & ASW0005rnb &  1.0  &  A,R   \\ 
SW50 & CFHTLS J140845+514913 &  212.1907 &   51.8205 &  0.7 & 19.7 &  1.5 &  2.7 & ASW0005o38 &  1.0  &  A,E   \\ 
SW51 & CFHTLS J141056+533225 &  212.7364 &   53.5405 &  0.6 & 19.4 &  1.7 &  2.0 & ASW0006kjx &  0.6  &  A,R/G   \\ 
SW52 & CFHTLS J141432+534004 &  213.6372 &   53.6679 &  0.7 & 21.4 &  0.9 &  2.0 & ASW0006jh5 &  1.0  &  A,R   \\ 
SW53 & CFHTLS J141448+545548 &  213.7010 &   54.9301 &  0.4 & 18.3 &  2.1 &  1.3 & ASW0006zc9 &  0.4  &  A,G   \\ 
SW54 & CFHTLS J141518+513915 &  213.8290 &   51.6542 &  0.4 & 18.3 &  3.0 &  1.7 & ASW00070vl &  1.0  &  D,E   \\ 
SW55 & CFHTLS J141927+533919 &  214.8627 &   53.6554 &  0.7 & 20.5 &  2.3 &  1.7 & ASW0006yrw &  0.5  &  A,R/G   \\ 
SW56 & CFHTLS J142432+550019 &  216.1354 &   55.0055 &  0.5 & 19.5 &  1.1 &  2.0 & ASW0007vx2 &  1.0  &  A,E   \\ 
SW57 & CFHTLS J142603+511421 &  216.5140 &   51.2393 &  --  &  --  &  4.4 &  2.7 & ASW0006mea &  1.0  &  A,G   \\ 
SW58 & CFHTLS J142620+561356 &  216.5870 &   56.2323 &  0.4 & 19.5 &  1.3 &  1.3 & ASW0007sez &  1.0  &  A/R,S   \\ 
SW59 & CFHTLS J142652+560001 &  216.7200 &   56.0004 &  --  &  --  &  1.5 &  1.7 & ASW0007t5y &  1.0  &  R,R   \\ 
SW60 & CFHTLS J142843+543713 &  217.1815 &   54.6204 &  0.4 & 19.7 &  1.3 &  1.3 & ASW0007pga &  0.9  &  D,D   \\ 
SW61 & CFHTLS J142934+562541 &  217.3926 &   56.4281 &  0.5 & 19.0 &  5.9 &  2.7 & ASW0009cjs &  1.0  &  A,G   \\ 
SW62 & CFHTLS J143055+572431 &  217.7333 &   57.4088 &  0.7 & 19.3 &  1.0 &  2.0 & ASW0007wfj &  1.0  &  A,R   \\ 
SW63 & CFHTLS J143100+564603 &  217.7510 &   56.7675 &  --  &  --  &  1.8 &  1.7 & ASW00086xq &  1.0  &  A,E   \\ 
SW64 & CFHTLS J143353+542310 &  218.4736 &   54.3862 &  0.8 & 19.8 &  1.6 &  2.0 & ASW0009cox &  0.9  &  A,R/G   \\ 
SW65 & CFHTLS J143454+522850 &  218.7270 &   52.4808 &  0.6 & 19.4 &  4.4 &  2.3 & ASW0007k4r &  0.4  &  Q,G/R   \\ 
SW66 & CFHTLS J143627+563832 &  219.1164 &   56.6425 &  0.5 & 19.5 &  1.5 &  2.3 & ASW0008swn &  1.0  &  A,D   \\ 
SW67 & CFHTLS J143631+571131 &  219.1315 &   57.1922 &  0.7 & 20.9 &  1.3 &  1.3 & ASW0008pag &  0.8  &  D/A,R   \\ 
SW68 & CFHTLS J143651+530705 &  219.2150 &   53.1183 &  0.6 & 19.2 &  3.1 &  1.3 & ASW0007h27 &  1.0  &  A,E/G   \\ 
SW69 & CFHTLS J143658+533807 &  219.2425 &   53.6355 &  0.7 & 19.6 &  0.9 &  2.7 & ASW0007hu2 &  0.9  &  D,D   \\ 
SW70 & CFHTLS J143838+572647 &  219.6589 &   57.4464 &  0.8 & 20.2 &  1.1 &  2.0 & ASW0008qsm &  1.0  &  A,R   \\ 
SW71 & CFHTLS J143950+544606 &  219.9610 &   54.7686 &  --  &  --  &  1.7 &  1.3 & ASW00085cp &  0.6  &  A,G/R   \\ 
SW72 & CFHTLS J220215+012124 &  330.5635 &    1.3567 &  0.3 & 17.4 &  4.6 &  1.7 & ASW0005qiz &  0.5  &  rA,G   \\ 
SW73 & CFHTLS J220256+023432 &  330.7370 &    2.5758 &  --  &  --  &  6.8 &  2.3 & ASW0007e08 &  1.0  &  A,G/C   \\ 
SW74 & CFHTLS J220722+013610 &  331.8441 &    1.6030 &  0.8 & 20.3 &  1.0 &  2.0 & ASW0001uuz &  0.6  &  A,R   \\ 
SW75 & CFHTLS J221101+003401 &  332.7582 &    0.5671 &  0.9 & 20.0 &  1.1 &  1.3 & ASW00090l5 &  1.0  &  A,E/R   \\ 
SW76 & CFHTLS J221306+014708 &  333.2758 &    1.7856 &  --  & 17.1 &  1.4 &  1.7 & ASW0008wmr &  1.0  &  A,S   \\ 
SW77 & CFHTLS J221513+010240 &  333.8060 &    1.0445 &  --  &  --  &  0.8 &  1.7 & ASW0008dxh &  0.6  &  A,R/G   \\ 
SW78 & CFHTLS J221519+005758 &  333.8321 &    0.9661 &  0.4 & 20.2 &  1.0 &  1.7 & ASW0008xbu &  1.0  &  A,D   \\ 
SW79 & CFHTLS J221716+015826 &  334.3189 &    1.9739 &  0.1 & 21.6 &  1.0 &  1.7 & ASW00096rm &  1.0  &  A/R,R   \\ 
SW80 & CFHTLS J222007-002505 &  335.0307 &   -0.4182 &  0.3 & 21.4 &  1.2 &  2.0 & ASW00094fq &  1.0  &  Q,R/G   \\ 
\end{longtable}
\end{center}


\twocolumn

\begin{figure*}
\begin{center}
\includegraphics[scale=1.5]{sw-cfhtls-figs/lenscandfin.pdf}
\caption{ \label{fig:lch1}
Sample of lens candidates with $R_{avg}>=1.3$.
}
\end{center}
\end{figure*}

\begin{figure*}
\begin{center}
\includegraphics[scale=1.5]{sw-cfhtls-figs/lenscandfin_1.pdf}
\caption{ \label{fig:lcl}
Sample of lens candidates $R_{avg}>=1.3$.
}
\end{center}
\end{figure*}


% % % % % % % % % % % % % % % % % % % % % % % % % % % % % % % % % % % % % % % 

\subsection{Recovery of known \cfhtls lenses with \sw}
\label{sec:results:known}

{\it Describe which lenses were recovered and explain any cases missed by
citizen scientists}

\begin{figure}
\begin{center}
\includegraphics[scale=1.0]{sw-cfhtls-figs/comp_reinst.pdf}
\caption{ \label{fig:compre} Completeness of simulated and known lens
samples based on the final \sw sample. }
\end{center}
\end{figure}


{\it Rejection rate. Completeness and purity at P > retirement, P > 95\%, and 
as function of probability P. 

Summarize performance at some fiducial threshold: eg P = 95\%.}

\subsection{Image separation distribution}
\label{sec:results:isd}


\begin{figure}
\begin{center}
\includegraphics[scale=1.0]{sw-cfhtls-figs/isd_cfhtls_sw.pdf}
\caption{ \label{fig:isd} Image separation distribution. Comparing
theoretical predictions with the \cfhtls known lens samples (REF) and
the same \cfhtls sample after combining with the incremental lens sample
from \sw }
\end{center}
\end{figure}


%%%%%%%%%%%%%%%%%%%%%%%%%%%%%%%%%%%%%%%%%%%%%%%%%%%%%%%%%%%%%%%%%%%%%%%%%%%%%%

\section{Discussion}
\label{sec:discuss}

{\it Differences with robots: what types of lenses are found at \sw?

Selection function. Missing system.

Further work.

Describe the importance of this kind of study 
- being able to quantify the completeness of the lens samples
- state how this will be used in a future \sw paper }

\begin{figure}
\begin{center}
\includegraphics[scale=1.0]{sw-cfhtls-figs/zlens_reinst.pdf}
\caption{ \label{fig:zlre}
Comparison of Lens redshift and Arc radius for all three lens samples,
namely, from \sw, SARCS and those from the RingFinder.}
\end{center}
\end{figure}

{\bf TBD: 
add section on offline analysis - extra lenses found etc;  
expand FP section;  
mention EGS moustakas and other lens samples; 
comment on any lensed quasars, on any exotic lenses;
check if some candidates were detected because they were hidden
underneath the sims ie. from the D11;

}
\subsection{Blind Lens Search}
We looked at the locations of simulated and real lenses from our data
those were missed by \sw compared to the locations of the lenses that
got detected. The real lens sample consists of a total of 383 candidates
which have $P_l>2.e-5$ and received a rank of $R_{avg}>0$ from the
expert. We do not find any obvious dependency in the rate of detections
as a function of the position of a given lens for both simulated and
real lens sample. Thus, the completeness of the lens sample is not
affected by whether a lens is located close to the border or well within
the center (see  Figure~\ref{fig:comppos}).

\begin{figure*}
\begin{center}
\includegraphics[scale=0.95]{sw-cfhtls-figs/completeness_pos.pdf}
\caption{ \label{fig:comppos}
Completeness as a function of location of lenses. Simulated lenses (left) and
real lens candidate (right) are shown. {\bf: TBD - Add a third color for
P$>$ Pthresh which will be defn of "accepted"} Irrespective of whether the
lenses are detected or rejected and whether they are simulated or real,
there is no obvious dependency on where the lenses are located. 
}
\end{center}
\end{figure*}

\subsection{Limitations and Caveats of the training sample}
The simulated sample has certain limitations due to lack of our understanding of
various phenonmena in the Universe and due to uncertainties in various
parameters of our model. Here, we describe some of the cases or aspects in which the
simulations are known to have failed or seem unrealistic. 

The parameters required by various scaling relations and the models primarily
depend on the photometry of the galaxies, groups and quasars detected in the
survey. For galaxy-scale lenses, the lensing galaxies at higher redshifts or
which are fainter have poor photometric measurements. This causes relatively
larger uncertainties in its luminosity and velocity dispersion and leads to
simulated lenses which look implausible. For example, due to a larger
uncertainty in the velocity dispersion of the lensing galaxy, the lensed images
may have larger image separation than what is expected given the visual priors
from the galaxy.

At group-scales, the photometric and redshift estimates are used when
defining the group membership. Therefore, errors in redshift estimates generate
galaxy groups with BGG or member galaxies with dissimilar properties. In some
cases, low redshift spiral galaxies are incorrectly assigned high redshift.
Spiral galaxies are typically less massive and low redshift spiral galaxies are
unlikely to act as gravitational lenses. Hence, the resulting simulated lenses
are not convincing.

We use single Sersic component to describe the light profiles of background
galaxies. This is clearly not the most accurate description for galaxies,
especially, star-forming galaxies which form a significant fraction of the
lensed galaxy population. Star-forming galaxies have complex structures such as
star forming knots, spiral arms, bars and disks. 

%%%%%%%%%%%%%%%%%%%%%%%%%%%%%%%%%%%%%%%%%%%%%%%%%%%%%%%%%%%%%%%%%%%%%%%%%%%%%%

\section{Summary and Conclusions}
\label{sec:conclude}
In this paper, we describe the framework and procedure used to generate
simulated lens sample for the blind lens search in the \cfhtls survey in
collaboration with \sw. The aim of this lens search is to find lenses that have
been missed by lens finding algorithms. The simulated lens sample is used for
training the citizen scientists, calibrating their performance and rejecting
unlikely lenses from the sample based on the classifications of citizen
scientists. As a result, the simulated lenses need to consist of realistic
looking lenses.

We use the photometric and redshift catalogues for the foreground galaxies and
additionally, color catalogues for background galaxies and quasars. We
use scaling relations to relate light properties to properties such as mass and
size to generate lens models. We further account for the
instrumental effects such as seeing and noise before creating the final lensed
images of model background sources. We add the lensed images on top of the real
galaxies and groups in the \cfhtls data in all filters.

We draw the following conclusions:

\begin{itemize} 

\item Crowd-sourced gravitational lens detection works, as shown in by comparing with real lenses in CFHTLS:
Real (robotically-detected and expert-confirmed) lenses are
recovered/missed at similar/comparable/different rates C’\% and P’\% - this is a partial test of supervised vs unsupervised learning

%Which real lenses were missed? False negatives

\item We found a sample of new gravitational lens candidates. An
expert-graded sample of 74 with 44 promising and 30 low probability
candiates.

The SW lenses are different from the robotically (RingFinder and ArcFinder) detected lenses, in the following ways.
XXXXX


\end{itemize}


%%%%%%%%%%%%%%%%%%%%%%%%%%%%%%%%%%%%%%%%%%%%%%%%%%%%%%%%%%%%%%%%%%%%%%%%
%%  ACKNOWLEDGMENTS
%%%%%%%%%%%%%%%%%%%%%%%%%%%%%%%%%%%%%%%%%%%%%%%%%%%%%%%%%%%%%%%%%%%%%%%%

\section*{Acknowledgements}
 
We thank all \Ncollaboration members of the \sw community for their
contributions to the project so far. A complete list of collaborators is
given at... In particular we would like to recognise the efforts of XXX,
YYY etc in moderating the discussion.

We are also grateful to Brooke Simmons, David Hogg, XXX and YYY for many useful
conversations about citizen science and gravitational lens detection. 
%
PJM was given support by the Royal Society, in the form of a research
fellowship, and by the U.S. Department of Energy under contract number DE-AC02-76SF00515.
%
AV acknowledges support from the Leverhulme Trust in the form of a research
fellowship.
%
The work of AM and SM was supported by World Premier International Research
Center Initiative (WPI Initiative), MEXT, Japan.
%
% Zooniverse / Sloan Foundation.
% 
% Other authors.
PJM and ES thank the Institute of Astronomy and Astrophysics, Academia Sinica
(ASIAA) and Taiwan's Ministry of Science and Technology (MOST) for their
financial support of the workshop ``Citizen Science in Astronomy'' in March
2014, at which some parts of the SWAP analysis was developed.


The \sw project is open source. 
The web app was developed at https://github.com/Zooniverse/Lens-Zoo
while the SWAP analysis software was developed at
https://github.com/drphilmarshall/SpaceWarps.

This work is based on observations obtained with MegaPrime/MegaCam, a joint
project of CFHT and CEA/IRFU, at the Canada-France-Hawaii Telescope (CFHT) which
is operated by the National Research Council (NRC) of Canada, the Institut
National des Sciences de l'Univers of the Centre National de la Recherche
Scientifique (CNRS) of France, and the University of Hawaii. This research used
the facilities of the Canadian Astronomy Data Centre operated by the National
Research Council of Canada with the support of the Canadian Space Agency.
CFHTLenS data processing was made possible thanks to significant computing
support from the NSERC Research Tools and Instruments grant program.

%%%%%%%%%%%%%%%%%%%%%%%%%%%%%%%%%%%%%%%%%%%%%%%%%%%%%%%%%%%%%%%%%%%%%%%%%%%%%%
%  APPENDICES
%%%%%%%%%%%%%%%%%%%%%%%%%%%%%%%%%%%%%%%%%%%%%%%%%%%%%%%%%%%%%%%%%%%%%%%%%%%%%%

\appendix


%%%%%%%%%%%%%%%%%%%%%%%%%%%%%%%%%%%%%%%%%%%%%%%%%%%%%%%%%%%%%%%%%%%%%%%%%%%%%%
%  REFERENCES
%%%%%%%%%%%%%%%%%%%%%%%%%%%%%%%%%%%%%%%%%%%%%%%%%%%%%%%%%%%%%%%%%%%%%%%%%%%%%%

% MNRAS does not use bibtex, input .bbl file instead. 
% Generate this in the makefile using bubble script in scriptutils:

% bubble -f paper-lcr.tex references.bib 
% \input{paper-lcr.bbl}

\bibliographystyle{apj}
\bibliography{references_cfhtls}
%\bibliography{references}


%%%%%%%%%%%%%%%%%%%%%%%%%%%%%%%%%%%%%%%%%%%%%%%%%%%%%%%%%%%%%%%%%%%%%%%%%%%%%%

\label{lastpage}
\bsp

\end{document}

%%%%%%%%%%%%%%%%%%%%%%%%%%%%%%%%%%%%%%%%%%%%%%%%%%%%%%%%%%%%%%%%%%%%%%%%%%%%%%
